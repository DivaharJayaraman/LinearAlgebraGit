\documentclass[12pt,a5paper,smallborder]{refrep}
%\IfFileExists{ajr.sty}{\usepackage{ajr}}{}
\usepackage[a5paper]{geometry}
\settextfraction{0.7}

\title{Script for videos
\footnote{Currently: use Monosnap (free); then ``Record Video''; 1048x622 area; medium quality; move self to top-left; keyboard viewer to bottom-right; by default saves to Picture\slash Monosnap folder.}}
\author{A~J Roberts}
\date{\today}

\begin{document}

\maketitle
\tableofcontents

\chapter{Vectors}


\section{Use Matlab/Octave for vector computation}

\begin{description}
\item[startQuitMatlab]
To start Matlab, find on your computer or web service the icon for Matlab which on my computer is over here.  
Click and Matlab will start.  
Starting takes five seconds or so, and while it starts up this classic image is displayed.  
Eventually this Matlab window appears and is usually divided into five panes that you see here.  
We will only use the middle pane.  
In the lower right corner I also show this keyboard to help show the keys I type.  
In the middle pane we later do some exciting calculations, but for now let's just test by typing \verb|1+1| followed by the enter\slash return key.  
Matlab tells us the answer \verb|2|.  
To finish the session with Matlab, just type \verb|quit| followed by the enter\slash return key.  
This finishes a first use of Matlab.



\item[startQuitOctave]
To start Octave, find on your computer or web service the icon for Octave which on my computer is over here.  
Click and Octave will start in a window like this.  
Let's make this window bigger so you can see it more clearly.
In the lower right corner I also show this keyboard to help show the keys I type.  
Later we do some exciting calculations with Octave, but for now let's just test by typing \verb|1+1| followed by the enter\slash return key.  
Octave tells us the answer \verb|2|.  
To finish the session with Octave, just type \verb|quit| followed by the enter\slash return key.  
This finishes a first use of Octave.


\end{description}





\chapter{Systems of linear equations}




\end{document}
