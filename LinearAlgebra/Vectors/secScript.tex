%!TEX root = ../larxxia.tex


\section{Use \script\ for vector computation}
\label{sec:umovc}
\secttoc

\index{Matlab|(}
\index{Octave|(}

\begin{quoted}{\index{Babbage, Charles}Charles Babbage, 1832}
It is the science of \emph{calculation},---which becomes continually more necessary at each step of our progress, and which must ultimately govern the whole of the applications of science to the arts of life.
\end{quoted}

Subsequent chapters invoke the computer packages \script\ to perform calculations that would be tedious and error prone when done by hand.
This section introduces \script\ so that you can start to become familiar with it on small problems.
You should directly compare the computed answer with your calculation by hand.
The aim is to develop some basic confidence with \script\ before later using it to save considerable time in longer tasks.

\begin{itemize}
\item  \script[1]\ is commercial software available from Mathworks.\footnote{\url{http://mathworks.com}}
It is also useable over the internet as \script[1]-Online or \script[1]-Mobile.
\item \script[2]\ is free software, that for our limited purposes is almost the same as \script[1], and downloadable over the internet.\footnote{\url{https://www.gnu.org/software/octave/}}
Octave is also freely useable over the internet.\footnote{\url{http://octave-online.net}} 
\item Alternatively, your home institution may provide \script\ via a web service that is useable from smart phones, tablets and computers.
\end{itemize}



\begin{comment}
Avoid plotting because I do not want to confuse a vector with a row vector of plot points.  Plotting is also not high priority.
\end{comment}




\begin{table}
\caption{Use \script\ to help compute vector results with the following basics.  This and subsequent tables throughout the book summarise \script\ for our use.
\index{Matlab@\textsc{Matlab}|textbf}\index{Octave|textbf}} \label{tbl:mtlbpre}
\hrule
\begin{minipage}{\linewidth}
\begin{itemize}
\item Real numbers are limited to being zero or of magnitude from \(10^{-323}\)  to~\(10^{+308}\), both positive and negative (called the \bfidx{floating point} numbers).
Real numbers are computed and stored to a maximum \idx{precision} of nearly sixteen \idx{significant digits}.%
\footnote{If desired, `\idx{computer algebra}' software provides us with an arbitrary level of precision, even exact.
Current computer algebra software includes the free Sage, Maxima and Reduce, and the commercial Maple, Mathematica and (via Matlab) MuPad.}

\item \script\ potentially uses complex numbers~(\CC), but mostly we stay within real numbers~(\RR).

\item Each \script\ command is usually typed on one line by itself.

\item \index{[\ldots]@\texttt{[\ldots]}}\verb|[ . ; . ; . ]| where each dot denotes a number, forms vectors in~\(\RR^3\) (or use newlines instead of the semi-colons).  
Use \(n\)~numbers separated by semi-colons for vectors in~\(\RR^n\).

\item \index{=@\texttt{=}}\verb|=| assigns the result of the expression on the right to the variable name on the left.

If the result of an expression is not explicitly assigned to a variable, then by default it is assigned to the variable~\verb|ans| (denoted by ``\index{ans@\texttt{ans}}\verb|ans =|'' in \script).

\item Variable names are alphanumeric starting with a letter.

\item \index{size()@\texttt{size()}}\verb|size(v)| returns the number of components of the vector (Definition~\ref{def:vecs}): if the vector~\vv\ is in \(\RR^m\), then \verb|size(v)| returns \(\begin{bmatrix} m&1 \end{bmatrix}\).

\item \index{norm()@\texttt{norm()}}\verb|norm(v)| computes the \idx{length}\slash \idx{magnitude} of the vector~\verb|v| (Definition~\ref{def:veclen}).

\item \index{+,-,*@\texttt{+,-,*}}\verb|+,-,*| is vector\slash scalar addition, subtraction, and multiplication, but only provided the sizes of the two vectors are the same.
Parentheses \index{()@\texttt{()}}\verb|()| control the order of operations.

\item \index{/@\texttt{/}}\verb|/x| divides a vector\slash scalar by a scalar~\(x\).
However, be warned that \verb|/v| for a vector~\vv\ typically gives strange results as \script\ interprets it to mean you want to approximately solve some linear equation.

\item \index{^@\texttt{\^}}\verb|x^y| for scalars~\(x\) and~\(y\) computes~\(x^y\).

\item \index{dot()@\texttt{dot()}}\verb|dot(u,v)| computes the \idx{dot product} of vectors~\verb|u| and~\verb|v| (Definition~\ref{def:dotprod})---if they have the same size.

\item \index{acos()@\texttt{acos()}}\verb|acos(q)| computes the \idx{arc-cos}, the \idx{inverse cosine}, of the scalar~\verb|q| in radians.  To find the angle in degrees use \verb|acos(q)*180/pi|\,.

\item \index{quit@\texttt{quit}}\verb|quit| terminates the \script\ session.
\end{itemize}
\end{minipage}
\hrule
\end{table}




\begin{example} \label{eg:}
Use the \script\ command \index{norm()@\texttt{norm()}}\verb|norm()| to compute the length\slash magnitude of the following vectors (Definition~\ref{def:veclen}).
\begin{enumerate}
\item \((2,-1)\)
\begin{solution} 
Start \script.  After a prompt, ``\verb|>>|'' in \script[1] or ``\verb|octave:1>|'' in \script[2], type a command, followed by the Return\slash Enter key to get it executed.
As indicated by Table~\ref{tbl:mtlbpre} the numbers with brackets separated by semi-colons forms a vector, and the \verb|=|~character assigns the result to variable for subsequent use.

\begin{tabular}{@{}*2{p{0.45\linewidth}}@{}}\raggedright
Assign the vector to a variable~\verb|a| by the command \verb|a=[2;-1]|.
Then executing \verb|norm(a)| reports \verb|ans = 2.2361| as shown in the dialogue to the right.
&\begin{verbatim}
>> a=[2;-1]
a =
   2
  -1
>> norm(a)
ans =  2.2361
\end{verbatim}
\end{tabular}

This computes the answer \(|(2,-1)|=\sqrt{2^2+(-1)^2}=\sqrt5=2.2361\) (to five significant digits which we take to be practically exact).

The \idx{qr-code} appearing in the margin here encodes these \script\ commands.  
\setbox\ajrqrbox\hbox{\qrcode{% vector length
a=[2;-1]
norm(a)
}}%
\marginpar{\usebox{\ajrqrbox\\[2ex]}}%
You may scan such qr-codes with your favourite app\footnote{At the time of writing, qr-code scanning applications for smart-phones include \emph{WaspScan} and \emph{QRReader}---but I have no expertise to assess their quality.}, and then copy and paste the code direct into a \script\ client.
Alternatively, if reading an electronic version of this book, then you may copy and paste the commands; however, be warned that the quote character~\verb|'| usually needs correcting.
Although here the saving in typing is negligible, later you can save considerable typing
\end{solution}

\item \((-1,1,-5,4)\)
\begin{solution} 
In \script:

\begin{tabular}{@{}*2{p{0.47\linewidth}}@{}}\raggedright
Assign the vector to a variable with \verb|b=[-1;1;-5;4]| as shown to the right.
Then execute \verb|norm(b)| and find that \script\ reports \verb|ans = 6.5574|
&\begin{verbatim}
>> b=[-1;1;-5;4]
b =
  -1
   1
  -5
   4
>> norm(b)
ans =  6.5574
\end{verbatim}
\end{tabular}
\setbox\ajrqrbox\hbox{\qrcode{% vector length
b=[-1;1;-5;4]
norm(b)
}}%
\marginpar{\usebox{\ajrqrbox\\[2ex]}}%

Hence \(6.5574\) is the length of~\((-1,1,-5,4)\) (to five significant digits which we take to be practically exact).
\end{solution}

\item \((-0.3,4.3,-2.5,-2.8,7,-1.9)\)
\begin{solution} 
In \script:
\begin{enumerate}
\item assign the vector with the command\\ 
\verb|c=[-0.3;4.3;-2.5;-2.8;7;-1.9]|
\item execute \verb|norm(c)| and find that \script\ reports \verb|ans = 9.2347|
\end{enumerate}
\setbox\ajrqrbox\hbox{\qrcode{% vector length
c=[-0.3;4.3;-2.5;-2.8;7;-1.9]
norm(c)
}}%
\marginpar{\usebox{\ajrqrbox\\[2ex]}}%
Hence the length of vector~\((-0.3,4.3,-2.5,-2.8,7,-1.9)\) is~\(9.2347\) (to five significant digits).
\end{solution}

\end{enumerate}
\end{example}






\begin{example} \label{eg:}
Use \script\ operators \index{+,-,*@\texttt{+,-,*}}\verb|+,-,*| to compute the value of the expressions \(\uv+\vv\), \(\uv-\vv\), \(3\uv\) for vectors \(\uv=(-4.1,1.7,4.1)\) and \(\vv=(2.9,0.9,-2.4)\) (Definition~\ref{def:vecops}).
\begin{solution}  In \script\ type the commands, each followed by Return\slash Enter key.
\setbox\ajrqrbox\hbox{\qrcode{% vector ops
u=[-4.1;1.7;4.1]
v=[2.9;0.9;-2.4]
u+v
u-v
3*u
}}%
\marginpar{\usebox{\ajrqrbox\\[2ex]}}% 

\begin{tabular}{@{}*2{p{0.47\linewidth}}@{}}\raggedright
Assign the named vectors with the commands \verb|u=[-4.1;1.7;4.1]| and \verb|v=[2.9;0.9;-2.4]| to see the two steps in the dialogue to the right.
&\begin{verbatim}
>> u=[-4.1;1.7;4.1]
u =
  -4.1000
   1.7000
   4.1000
>> v=[2.9;0.9;-2.4]
v =
   2.9000
   0.9000
  -2.4000
\end{verbatim}
\end{tabular}

\begin{tabular}{@{}*2{p{0.47\linewidth}}@{}}\raggedright
Execute \verb|u+v| to find from the dialogue on the right that the sum \(\uv+\vv=(-1.2,2.6,1.7)\).
&\begin{verbatim}
>> u+v
ans =
  -1.2000
   2.6000
   1.7000
\end{verbatim}
\end{tabular}

\begin{tabular}{@{}*2{p{0.47\linewidth}}@{}}\raggedright
Execute \verb|u-v| to find from the dialogue on the right that the difference \(\uv-\vv=(-7,0.8,6.5)\).
&\begin{verbatim}
>> u-v
ans =
  -7.0000
   0.8000
   6.5000
\end{verbatim}
\end{tabular}

\begin{tabular}{@{}*2{p{0.47\linewidth}}@{}}\raggedright
Execute \verb|3*u| to find from the dialogue on the right that the scalar multiple \(3\uv=(-12.3,5.1,12.3)\)  (the asterisk is essential to compute multiplication).
&\begin{verbatim}
>> 3*u
ans =
  -12.3000
    5.1000
   12.3000
\end{verbatim}
\end{tabular}

\end{solution}
\end{example}




\begin{example} \label{eg:}
Use \script\ to confirm that \(2(2\pv-3\qv)+6(\qv-\pv)=-2\pv\) for vectors \(\pv=(1,0,2,-6)\) and \(\qv=(2,4,3,5)\).
\begin{solution} 
In \script\

\begin{tabular}{@{}*2{p{0.47\linewidth}}@{}}\raggedright
Assign the first vector with \verb|p=[1;0;2;-6]| as shown to the right.
&\begin{verbatim}
>> p=[1;0;2;-6]
p =
   1
   0
   2
  -6
\end{verbatim}
\end{tabular}
\setbox\ajrqrbox\hbox{\qrcode{% vector ops
p=[1;0;2;-6]
q=[2;4;3;5]
2*(2*p-3*q)+6*(q-p)
ans+2*p
}}%
\marginpar{\usebox{\ajrqrbox\\[2ex]}}% 

\begin{tabular}{@{}*2{p{0.47\linewidth}}@{}}\raggedright
Assign the other vector with \verb|q=[2;4;3;5]|.
&\begin{verbatim}
>> q=[2;4;3;5]
q =
   2
   4
   3
   5
\end{verbatim}
\end{tabular}

\begin{tabular}{@{}*2{p{0.47\linewidth}}@{}}\raggedright
Compute \(2(2\pv-3\qv)+6(\qv-\pv)\) with the command \verb|2*(2*p-3*q)+6*(q-p)| as shown to the right, and see the result is evidently~\(-2\pv\).
&\begin{verbatim}
>> 2*(2*p-3*q)+6*(q-p)
ans =
   -2
    0
   -4
   12
\end{verbatim}
\end{tabular}

\begin{tabular}{@{}*2{p{0.47\linewidth}}@{}}\raggedright
Confirm it is~\(-2\pv\) by adding~\(2\pv\) to the above result with the command \verb|ans+2*p| as shown to the right, and see the zero vector result.
&\begin{verbatim}
>> ans+2*p
ans =
   0
   0
   0
   0
\end{verbatim}
\end{tabular}

\end{solution}
\end{example}




\begin{example} \label{eg:}
Use \script\ to confirm the commutative law (Theorem~\ref{thm:vecopsa}) \(\uv+\vv=\vv+\uv\) for vectors
\(\uv=(8,-6,-4,-2)\) and \(\vv=(4,3,-1)\).
\begin{solution} 
In \script\

\begin{tabular}{@{}*2{p{0.47\linewidth}}@{}}\raggedright
Assign \(\uv=(8,-6,-4,-2)\)  with command \verb|u=[8;-6;-4;-2]| as shown to the right.
&\begin{verbatim}
>> u=[8;-6;-4;-2]
u =
   8
  -6
  -4
  -2
\end{verbatim}
\end{tabular}
\setbox\ajrqrbox\hbox{\qrcode{% bad vectors
u=[8;-6;-4;-2]
v=[4;3;-1]
u+v
size(u)
size(v)
}}%
\marginpar{\usebox{\ajrqrbox\\[2ex]}}% 

\begin{tabular}{@{}*2{p{0.47\linewidth}}@{}}\raggedright
Assign \(\vv=(4,3,-1)\) with the command \verb|v=[4;3;-1]|.
&\begin{verbatim}
>> v=[4;3;-1]
v =
   4
   3
  -1
\end{verbatim}
\end{tabular}

\index{Error using@\texttt{Error using}}
\begin{tabular}{@{}*2{p{0.47\linewidth}}@{}}\raggedright
Compute \(\uv+\vv\) with the command \verb|u+v| as shown to the right.
\script[1] prints an error message because the vectors~\uv\ and~\vv\ are of different sizes and so cannot be added together.
&\begin{verbatim}
>> u+v
Error using  + 
Matrix dimensions must agree. 
\end{verbatim}
\end{tabular}

\begin{tabular}{@{}*2{p{0.47\linewidth}}@{}}\raggedright
Check the sizes of the vectors in the sum using \verb|size(u)| and \verb|size(v)| to confirm \(\uv\)~is in~\(\RR^4\) whereas \(\vv\)~is in~\(\RR^3\).
Hence the two vectors cannot be added (Definition~\ref{def:vecops}).
&\begin{verbatim}
>> size(u)
ans =
   4   1
>> size(v)
ans =
   3   1 
\end{verbatim}
\end{tabular}

\index{error:@\texttt{error:}}
Alternatively, \script[2] gives the following error message in which ``\idx{nonconformant arguments}'' means of wrong sizes.
\begin{verbatim}
error: operator +: nonconformant arguments 
(op1 is 4x1, op2 is 3x1)
\end{verbatim}
\end{solution}
\end{example}




\begin{activity}
You enter the two vectors into \script[1]\ by typing \verb|u=[1.1;3.7;-4.5]| and \verb|v=[1.7;0.6;-2.6]|.
\begin{itemize}
\item Which of the following is the result of typing the command \verb|u-v|?
\begin{parts} 
\item \tt\begin{tabular}r
    2.8000\\
    4.3000\\
   -7.1000
\end{tabular}
\item \tt\begin{tabular}r
   -0.6000\\
    3.1000\\
   -1.9000
\end{tabular}
\item \tt\begin{tabular}{p{9em}}
Error using  * \\
Inner matrix dimensions must agree. 
\end{tabular}
\item \tt\begin{tabular}r
    2.2000\\
    7.4000\\
   -9.0000
\end{tabular}
\end{parts}
\item Which is the result of typing the command \verb|2*u|?
\item Which is the result of typing the command \verb|u*v|?
\end{itemize}
\end{activity}






\begin{example} \label{eg:}
Use \script\ to compute the angles between the pair of vectors \((4,3)\) and \((5,12)\) (Theorem~\ref{thm:anglev}).
\begin{solution} In \script

\begin{tabular}{@{}p{0.34\linewidth}p{0.6\linewidth}@{}}\raggedright
Because each vector is used twice in the formula \(\cos\theta=(\uv\cdot\vv)/(|\uv||\vv|)\), give each a name as shown to the right.
&\begin{verbatim}
>> u=[4;3]
u =
   4
   3
>> v=[5;12]
v =
    5
   12
\end{verbatim}
\end{tabular}
\setbox\ajrqrbox\hbox{\qrcode{% vector angle
u=[4;3]
v=[5;12]
cost=dot(u,v)/norm(u)/norm(v)
theta=acos(cost)*180/pi
}}%
\marginpar{\usebox{\ajrqrbox\\[2ex]}}% 

\begin{tabular}{@{}p{0.34\linewidth}p{0.6\linewidth}@{}}\raggedright
Then invoke \verb|dot()| to compute the dot product in the formula for the cosine of the angle.
&\begin{verbatim}
>> cost=dot(u,v)/norm(u)/norm(v)
cost =  0.8615
\end{verbatim}
\end{tabular}

\begin{tabular}{@{}p{0.34\linewidth}p{0.6\linewidth}@{}}\raggedright
Lastly, invoke \verb|acos()| for the arc-cosine, then convert the radians to degrees to find the angle \(\theta=30.510^\circ\).
&\begin{verbatim}
>> theta=acos(cost)*180/pi
theta =  30.510
\end{verbatim}
\end{tabular}

\end{solution}
\end{example}




\begin{example} \label{eg:}
Verify the distributive law for the dot product \((\uv+\vv)\cdot\wv=\uv\cdot\wv+\vv\cdot\wv\) (Theorem~\ref{thm:dotopsf}) for vectors
\(\uv=(-0.1,-3.1,-2.9,-1.3)\), \(\vv=(-3,0.5,6.4,-0.9)\) and \(\wv=(-1.5,-0.2,0.4,-3.1)\).
\begin{solution} 
In \script\

\begin{tabular}{@{}*2{p{0.47\linewidth}}@{}}\raggedright
Assign vector \(\uv=(-0.1,-3.1,-2.9,-1.3)\)  with the command \verb|u=[-0.1;-3.1;-2.9;-1.3]| as shown to the right.
&\begin{verbatim}
>> u=[-0.1;-3.1;-2.9;-1.3]
u =
  -0.1000
  -3.1000
  -2.9000
  -1.3000
\end{verbatim}
\end{tabular}
\setbox\ajrqrbox\hbox{\qrcode{% dot distributive
u=[-0.1;-3.1;-2.9;-1.3]
v=[-3;0.5;6.4;-0.9]
w=[-1.5;-0.2;0.4;-3.1]
dot(u+v,w)
dot(u,w)+dot(v,w)
}}%
\marginpar{\usebox{\ajrqrbox\\[2ex]}}% 

\begin{tabular}{@{}*2{p{0.47\linewidth}}@{}}\raggedright
Assign vector \(\vv=(-3,0.5,6.4,-0.9)\)  with the command \verb|v=[-3;0.5;6.4;-0.9]| as shown to the right.
&\begin{verbatim}
>> v=[-3;0.5;6.4;-0.9]
v =
  -3.0000
   0.5000
   6.4000
  -0.9000
\end{verbatim}
\end{tabular}

\begin{tabular}{@{}*2{p{0.47\linewidth}}@{}}\raggedright
Assign vector \(\wv=(-1.5,-0.2,0.4,-3.1)\)  with the command \verb|w=[-1.5;-0.2;0.4;-3.1]| as shown to the right.
&\begin{verbatim}
>> w=[-1.5;-0.2;0.4;-3.1]
w =
  -1.5000
  -0.2000
   0.4000
  -3.1000
\end{verbatim}
\end{tabular}

\begin{tabular}{@{}*2{p{0.47\linewidth}}@{}}\raggedright
Compute the dot product \((\uv+\vv)\cdot\wv\)  with the command \verb|dot(u+v,w)| to find the answer is~\(13.390\).
&\begin{verbatim}
>> dot(u+v,w)
ans =  13.390
\end{verbatim}
\end{tabular}

\begin{tabular}{@{}*2{p{0.47\linewidth}}@{}}\raggedright
Compare this with the dot product expression \(\uv\cdot\wv+\vv\cdot\wv\)  via the command \verb|dot(u,w)+dot(v,w)| to find the answer is~\(13.390\).
&\begin{verbatim}
>> dot(u,w)+dot(v,w)
ans =  13.390
\end{verbatim}
\end{tabular}

That the two answers agree verifies the distributive law for dot products.
\end{solution}
\end{example}



\begin{activity}
Given two vectors~\uv\ and~\vv\ that have already been typed into \script,
which of the following expressions could check the identity that \((\uv-2\vv)\cdot(\uv+\vv)=\uv\cdot\uv-\uv\cdot\vv-2\vv\cdot\vv\)\,?
\begin{enumerate}
\item \verb|(u-2*v)*(u+v)-u*u+u*v+2*v*v|
\item \verb|dot(u-2*v,u+v)-dot(u,u)+dot(u,v)+2*dot(v,v)|
\item \verb|dot(u-2v,u+v)-dot(u,u)+dot(u,v)+2dot(v,v)|
\item None of the others.
\end{enumerate}
\end{activity}




Many other books \cite[\S\S1.1--3, e.g.]{Quarteroni2006} give more details about the basics than the essentials introduced here.




\begin{quoted}{Charles Babbage}\index{Babbage, Charles}%
On two occasions I have been asked [by members of 
Parliament!], ``Pray, Mr.~Babbage, if you put into the machine wrong figures, will the right answers come out?''

I am not able rightly to apprehend the kind of confusion of ideas that could provoke such a question.
\end{quoted}


\index{Matlab|)}
\index{Octave|)}


\subsection{Exercises}


\begin{exercise} \label{ex:} 
Use \script\ to compute the length of each of the following vectors (the first five have integer lengths).
\begin{parts}[6em]
\item \(\begin{bmatrix} 2\\3\\6 \end{bmatrix}\)
\answer{\(7\)}
\item \(\begin{bmatrix} 4\\-4\\7 \end{bmatrix}\)
\answer{\(9\)}
\item \(\begin{bmatrix} -2\\6\\9 \end{bmatrix}\)
\answer{\(11\)}
\item \(\begin{bmatrix} 0.5\\0.1\\-0.5\\0.7 \end{bmatrix}\)
\answer{\(1\)}
\item \(\begin{bmatrix} 8\\-4\\5\\-8 \end{bmatrix}\)
\answer{\(13\)}
\item \(\begin{bmatrix} 1.1\\1.7\\-4.2\\-3.8\\0.9 \end{bmatrix}\)
\answer{\(6.0819\)}
\item \(\begin{bmatrix} 2.6\\-0.1\\3.2\\-0.6\\-0.2 \end{bmatrix}\)
\answer{\(4.1725\)}
\item \(\begin{bmatrix} 1.6\\-1.1\\-1.4\\2.3\\-1.6 \end{bmatrix}\)
\answer{\(3.6851\)}
\end{parts}
\end{exercise}



\begin{exercise} \label{ex:} 
Use \script\ to determine which are wrong out of the following identities and relations for vectors \(\pv=(0.8,-0.3,1.1,2.6,0.1)\) and \(\qv=(1,2.8,1.2,2.3,2.3)\).
\begin{parts}
\item \(3(\pv-\qv)=3\pv-3\qv\)
\item \(2(\pv-3\qv)+3(2\qv-\pv)=\pv\)
\item \(\frac12(\pv-\qv)+\frac12(\pv+\qv)=\pv\)
\item \(|\pv+\qv|\leq|\pv|+|\qv|\)
\item \(|\pv-\qv|\leq|\pv|+|\qv|\)
\item \(|\pv\cdot\qv|\leq|\pv||\qv|\)
\end{parts}
\end{exercise}





\begin{exercise} \label{ex:} 
Use \script\ to find the angles between pairs of vectors in each of the following groups.
\begin{enumerate}\sloppy
\item \(\pv=(2,3,6)\), \(\qv=(6,2,-3)\), \(\rv=(3,-6,2)\)
\answer{They are all orthogonal.}

\item \(\uv=\begin{bmatrix}-1\\-7\\-1\\-7\end{bmatrix}\), \(\vv=\begin{bmatrix}-1\\4\\4\\4\end{bmatrix}\), \(\wv=\begin{bmatrix}1\\-4\\-4\\-4\end{bmatrix}\)
\setbox\ajrqrbox\hbox{\qrcode{%
u=[-1;-7;-1;-7]
v=[-1;4;4;4]
w=[1;-4;-4;-4]}}%
\marginpar{\usebox{\ajrqrbox\\[2ex]}}%
\answer{\(\theta_{uv}=147.44^\circ\), \(\theta_{uw}=32.56^\circ\), \(\theta_{vw}=180^\circ\)}

\item \(\uv=\begin{bmatrix}5\\1\\-1\\3\end{bmatrix}\), \(\vv=\begin{bmatrix}-6\\4\\2\\-5\end{bmatrix}\), \(\wv=\begin{bmatrix}-4\\2\\1\\-2\end{bmatrix}\)
\setbox\ajrqrbox\hbox{\qrcode{%
u=[5;1;-1;3]
v=[-6;4;2;-5]
w=[-4;2;1;-2]}}%
\marginpar{\usebox{\ajrqrbox\\[2ex]}}%
\answer{\(\theta_{uv}=142.78^\circ\), \(\theta_{uw}=146.44^\circ\), \(\theta_{vw}=12.10^\circ\)}

\item \(\uv=\begin{bmatrix}-9\\-8\\4\\8\end{bmatrix}\), \(\vv=\begin{bmatrix}9\\3\\1\\-3\end{bmatrix}\), \(\wv=\begin{bmatrix}-6\\5\\-2\\-4\end{bmatrix}\)
\setbox\ajrqrbox\hbox{\qrcode{%
u=[-9;-8;4;8]
v=[9;3;1;-3]
w=[-6;5;-2;-4]}}%
\marginpar{\usebox{\ajrqrbox\\[2ex]}}%
\answer{\(\theta_{uv}=146.44^\circ\), \(\theta_{uw}=101.10^\circ\), \(\theta_{vw}=108.80^\circ\)}

\item \(\av=\begin{bmatrix}-4.1\\9.8\\0.3\\1.4\\2.7\end{bmatrix}\), \(\bv=\begin{bmatrix}-0.6\\2.6\\-1.2\\-0.2\\-0.9\end{bmatrix}\), \(\cv=\begin{bmatrix}-2.8\\-0.9\\-6.2\\-2.3\\-4.7\end{bmatrix}\), \(\dv=\begin{bmatrix}1.8\\-3.4\\-8.6\\1.4\\1.8\end{bmatrix}\)
\setbox\ajrqrbox\hbox{\qrcode{%
a=[-4.1;9.8;0.3;1.4;2.7]
b=[-0.6;2.6;-1.2;-0.2;-0.9]
c=[-2.8;-0.9;-6.2;-2.3;-4.7]
d=[1.8;-3.4;-8.6;1.4;1.8]
}}%
\marginpar{\usebox{\ajrqrbox\\[2ex]}}%
\answer{\(\theta_{ab}=42.82^\circ\), \(\theta_{ac}=99.11^\circ\), \(\theta_{ad}=109.89^\circ\), \(\theta_{bc}=64.32^\circ\), \(\theta_{bd}=92.89^\circ\), \(\theta_{cd}=61.70^\circ\)}


\item \(\av=\begin{bmatrix}-0.5\\2.0\\-3.4\\1.8\\0.1\end{bmatrix}\), \(\bv=\begin{bmatrix}5.4\\7.4\\0.5\\0.7\\1.3\end{bmatrix}\), \(\cv=\begin{bmatrix} -0.2\\-1.5\\-0.3\\1.1\\2.5\end{bmatrix}\), \(\dv=\begin{bmatrix}1.0\\2.0\\-1.3\\-4.4\\-2.0\end{bmatrix}\)
\setbox\ajrqrbox\hbox{\qrcode{%
a=[-0.5;2.0;-3.4;1.8;0.1]
b=[5.4;7.4;0.5;0.7;1.3]
c=[ -0.2;-1.5;-0.3;1.1;2.5]
d=[1.0;2.0;-1.3;-4.4;-2.0]
}}%
\marginpar{\usebox{\ajrqrbox\\[2ex]}}%
\answer{\(\theta_{ab}=73.11^\circ\), \(\theta_{ac}=88.54^\circ\), \(\theta_{ad}=90.48^\circ\), \(\theta_{bc}=106.56^\circ\), \(\theta_{bd}=74.20^\circ\), \(\theta_{cd}=137.36^\circ\)}

\end{enumerate}
\end{exercise}





\begin{comment}%{ED498555.pdf}
why, what caused X?
how did X occur?
what-if? what-if-not?
how does X compare with Y?
what is the evidence for X?
why is X important?
\end{comment}



