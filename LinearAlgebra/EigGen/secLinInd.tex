%!TEX root = ../larxxia.tex

\section{Linear independent vectors may form a basis}
\label{sec:lisb}
\secttoc
\begin{comment}
\pooliv{p.92--7,198--208} \holti{\S2.3}
\end{comment}

\index{linearly dependent|(}
\index{linearly independent|(}
\index{linear dependence|(}
\index{linear independence|(}


In Chapter~\ref{ch:eesm} on symmetric matrices, the \idx{eigenvector}s from distinct \idx{eigenvalue}s are orthogonal.  
For general matrices the eigenvectors are not orthogonal---as introduced at the start of this Chapter~\ref{ch:gee}.  
But the orthogonal property is extremely useful.
Question: is there a similarly useful analogue of orthogonality?
Answer: yes; we now extend ``orthogonal'' with the more general concept of ``linear independence'' which for problems with general matrices replaces orthonormality.

One reason that orthogonal vectors are useful is that they can form an orthonormal basis and hence act as the unit vectors of an orthogonal coordinate systems.
The concept of linear independence is closely connected to coordinate systems which are not orthogonal.

\paragraph{Subspace coordinate systems} In any given problem we want two things from a \idx{general solution}: 
\begin{itemize}
\item firstly, the general solution must encompass every possibility (the solution must span the possibilities); and 
\item secondly, each possible solution should have a unique algebraic description in the general solution.
\end{itemize}
For an example of the need for a unique algebraic form, let's suppose we wanted to find solutions to the differential equation \(d^2y/dt^2-y=0\)\,. 
You might find \(y=3e^x+2e^{-x}\), whereas I find \(y=5\cosh x+\sinh x\), and a friend finds \(y=e^x+4\cosh x\).
By looking at these disparate algebraic forms it is apparent that we all disagree.
Should we all go and search for errors in the solution process?  No.
The reason is that all these solutions are the same.
The apparent differences arise only because you choose exponentials to represent the solution, whereas I choose hyperbolic functions: the solutions are the same, it is only the algebraic representation that appears different. 
In general, when we cannot immediately distinguish identical solutions, all algebraic manipulation becomes immensely more difficult due to algebraic ambiguity.
To avoid such immense difficulties, in both calculus and linear algebra, we need to introduce the concept of linear independence.

Linear independence empowers us, often implicitly, to use a non-orthogonal coordinate system in a subspace.
We replace the orthonormal standard unit vectors by any suitable set of basis vectors.
\marginpar{\begin{tikzpicture}
\newcommand{\ppoint}[2]{
    \pgfmathparse{#1*3+#2*1}\let\h\pgfmathresult
    \pgfmathparse{#1*1+#2*2}\let\v\pgfmathresult
    \addplot[red,mark=*,only marks] coordinates {(\h,\v)};
    \edef\tempe{\noexpand
    \node[left] at (axis cs:\h,\v) {$(#1,#2)$};
    }\tempe
    }
\begin{axis}[footnotesize,font=\footnotesize
  ,axis lines=none%,xlabel={$x$},ylabel={$y$}
  , axis equal image
  , view={0}{90}
  ,xmax=7.5,ymax=5.5,xmin=-7.5,ymin=-5.5
  ]
\addplot3[mesh,brown,samples=17,domain=-4:4,dotted] (3*x+y,x+2*y,0);
\addplot3[mesh,brown,samples=9,domain=-4:4] (3*x+y,x+2*y,0);
\addplot[blue,quiver={u=3,v=1},-stealth,thick] coordinates {(0,0)};
\node[right] at (axis cs:3,1) {$\vv_1$};
\addplot[blue,quiver={u=1,v=2},-stealth,thick] coordinates {(0,0)};
\node[above] at (axis cs:1,2) {$\vv_2$};
\ppoint21
\ppoint1{-3}
\ppoint{-2}3
\ppoint00
\end{axis}
\end{tikzpicture}}%
For example, in the plane any two vectors at an angle to each other suffice to be able to describe uniquely every vector (point) in the plane.
As illustrated in the margin, every point in the plane (end point of a vector) is a unique linear combination of the two basis vectors~\(\vv_1\) and~\(\vv_2\).
Such a pair of basis vectors, termed linearly independent, avoid the immense difficulties of algebraic ambiguity.



\subsection{Linearly (in)dependent sets}

This section defines linear (in)dependence, and then relates the concept to homogeneous linear equations, orthogonality, and sets of eigenvectors.

\begin{example}[2D non-orthogonal coordinates] \label{eg:2dnoc}
Show that every vector in the plane~\(\RR^2\) can be written uniquely as a \idx{linear combination} of the two vectors \(\vv_1=(2,1)\) and \(\vv_2=(1,2)\) that are shown in the margin.
\def\temp#1{\begin{tikzpicture}
\begin{axis}[footnotesize,font=\footnotesize
  ,axis lines=middle, axis equal image
  ,xmax=2.3,ymax=2.3
  ]
\addplot[blue,quiver={u=2,v=1},-stealth,thick] coordinates {(0,0)};
\node[below] at (axis cs:2,1) {$\vv_1$};
\addplot[blue,quiver={u=1,v=2},-stealth,thick] coordinates {(0,0)};
\node[right] at (axis cs:1,2) {$\vv_2$};
\ifcase#1
\or%1
\addplot[brown,quiver={u=-1.333,v=-0.667},-stealth] coordinates {(0,0)};
\addplot[brown,quiver={u=1.333,v=2.667},-stealth] coordinates {(-1.333,-0.667)};
\addplot[red,quiver={u=0,v=2},-stealth,thick] coordinates {(0,0)};
\node[right] at (axis cs:0,2) {$(0,2)$};
\or%2
\addplot[brown,quiver={u=1.333,v=0.667},-stealth] coordinates {(0,0)};
\addplot[brown,quiver={u=0.667,v=1.333},-stealth] coordinates {(1.333,0.667)};
\addplot[red,quiver={u=2,v=2},-stealth,thick] coordinates {(0,0)};
\node[left] at (axis cs:2,2) {$(2,2)$};
\or%3
\addplot[brown,quiver={u=2,v=1},-stealth] coordinates {(0,0)};
\addplot[brown,quiver={u=-1,v=-2},-stealth] coordinates {(2,1)};
\addplot[red,quiver={u=1,v=-1},-stealth,thick] coordinates {(0,0)};
\node[above] at (axis cs:1,-1) {\quad\qquad$(1,-1)$};
\or%4
\addplot[brown,quiver={u=-2,v=-1},-stealth] coordinates {(0,0)};
\addplot[brown,quiver={u=-1,v=-2},-stealth] coordinates {(-2,-1)};
\addplot[red,quiver={u=-3,v=-3},-stealth,thick] coordinates {(0,0)};
\node[above] at (axis cs:-3,-3) {\qquad\qquad$(-3,-3)$};
\else
\fi
\end{axis}
\end{tikzpicture}}%
\marginpar{\temp0}
\begin{solution} 
Let's start with some specific example vectors.
\begin{parts}
\item The vector \((0,2)\) may be written as the linear combination
\((0,2)=-\frac23\vv_1+\frac43\vv_2\) as shown. \temp1

\item The vector \((2,2)\) may be written as the linear combination
\((2,2)=\frac23\vv_1+\frac23\vv_2\) as shown. \temp2

\item The vector \((1,-1)\) may be written as the linear combination
\((1,-1)=\vv_1-\vv_2\) as shown. \temp3

\item The vector \((-3,-3)\) may be written as the linear combination
\((-3,-3)=-\vv_1-\vv_2\)\,. \temp4

\end{parts}

Now proceed to consider a general vector \((x,y)\) and seek it as a linear combination of~\(\vv_1\) and~\(\vv_2\), namely \((x,y)=c_1\vv_1+c_2\vv_2\)\,.
That is, let's write each and every point in the plane as a linear combination of~\(\vv_1\) and~\(\vv_2\) as illustrated in the margin.
\marginpar{\begin{tikzpicture}
\begin{axis}[small,font=\footnotesize
  ,axis lines=middle, axis equal image, view={0}{90}
  ,xmax=5.5,ymax=5.5,xmin=-5.5,ymin=-5.5
  ,xlabel={$x$},ylabel={$y$}
  ]
\addplot3[mesh,brown,samples=13,domain=-3:3,dotted] (2*x+y,x+2*y,0);
\addplot3[mesh,brown,samples=7,domain=-3:3] (2*x+y,x+2*y,0);
\addplot[blue,quiver={u=2,v=1},-stealth,thick] coordinates {(0,0)};
\node[right] at (axis cs:2,1) {$\vv_1$};
\addplot[blue,quiver={u=1,v=2},-stealth,thick] coordinates {(0,0)};
\node[above] at (axis cs:1,2) {$\vv_2$};
\end{axis}
\end{tikzpicture}}%
Rewrite the equation in matrix-vector form as
\begin{equation*}
\begin{bmatrix} \vv_1&\vv_2 \end{bmatrix}
\begin{bmatrix} c_1\\c_2 \end{bmatrix}
=\begin{bmatrix} x\\y \end{bmatrix}
,\quad\text{that is, }
V\cv=\begin{bmatrix} x\\y \end{bmatrix}
\quad\text{for }V=\begin{bmatrix} 2&1\\1&2 \end{bmatrix}.
\end{equation*}
For any given \((x,y)\), \(V\cv=(x,y)\) is a system of linear equations for the coefficients~\cv.
Theorem~\ref{thm:ftim2} asserts the system has a unique solution~\cv\ if and only iff the matrix~\(V\) is invertible.
Here the unique solution is then that the vector of coefficients
\begin{equation*}
\cv=V^{-1}\begin{bmatrix} x\\y \end{bmatrix}
=\begin{bmatrix} \frac23&-\frac13\\-\frac13&\frac23 \end{bmatrix}
\begin{bmatrix} x\\y \end{bmatrix}.
\end{equation*}
Equivalently, Theorem~\ref{thm:ftim2iii} asserts the system has a unique solution~\cv---unique coefficients~\cv---if and only iff the {homogeneous} system \(V\cv=\ov\) has only the zero solution \(\cv=\ov\)\,.
It is this last statement that leads to the upcoming Definition~\ref{def:lindep} of \idx{linear independence}.
\end{solution}
\end{example}


\begin{example}[3D failure] \label{eg:}
Show that vectors in~\(\RR^3\) are not written uniquely as a \idx{linear combination} of \(\vv_1=(-1,1,0)\), \(\vv_2=(1,-2,1)\) and \(\vv_3=(0,1,-1)\). 

One reason for the failure is that these three vectors only span a plane, as shown below in stereo.  The solution here looks at the different issue of unique representation.
\begin{center}
\qview{113}{117}{\begin{tikzpicture}
\begin{axis}[small,font=\small
  ,xlabel={$x$}, ylabel={$y$}, zlabel={$z$},label shift={-2ex}
  ,no marks ,domain=-2.5:2.5,axis equal image,zmax=2.5,zmin=-2.5
  ,view={\q}{30}
  ] 
\addplot3[surf,blue,samples=3,opacity=0.3]  {-x-y};
\addplot3[blue,thick,quiver={u=-1,v=1,w=0},-stealth] coordinates {(0,0,0)};
\node[right] at (axis cs:-1,1,0) {$\vv_1$};
\addplot3[blue,thick,quiver={u=1,v=-2,w=1},-stealth] coordinates {(0,0,0)};
\node[below] at (axis cs:1,-2,1) {$\vv_2$};
\addplot3[blue,thick,quiver={u=0,v=1,w=-1},-stealth] coordinates {(0,0,0)};
\node[below] at (axis cs:0,1,-1) {$\vv_3$};
\end{axis}
\end{tikzpicture}}
\end{center}
\begin{solution} 
As one example, consider the vector~\((1,0,-1)\):
\begin{eqnarray*}
(1,0,-1)&=&-1\vv_1+0\vv_2+1\vv_3\,;
\\(1,0,-1)&=&1\vv_1+2\vv_2+3\vv_3\,;
\\(1,0,-1)&=&-2\vv_1-1\vv_2+0\vv_3\,;
\\(1,0,-1)&=&(-1+t)\vv_1+t\vv_2+(1+t)\vv_3\,,\quad\text{for any }t.
\end{eqnarray*}
This last combination shows there an infinite number of ways to write \((1,0,-1)\) as a linear combination of~\(\vv_1\), \(\vv_2\) and~\(\vv_3\).
Such an infinity of linear combinations means that~\(\vv_1\), \(\vv_2\) and~\(\vv_3\) do not form a useful `coordinate system' because we cannot easily distinguish between the different combinations all giving the same vector.
The reason for the infinity of combinations is that there is a nontrivial linear combination of~\(\vv_1\), \(\vv_2\) and~\(\vv_3\) which is zero, namely \(\vv_1+\vv_2+\vv_3=\ov\)\,.
It is this last statement that leads to the Definition~\ref{def:lindep} of \idx{linear dependence}.
\end{solution}
\end{example}





\begin{definition} \label{def:lindep} 
A set of vectors \(\{\hlist\vv k\}\) is \bfidx{linearly dependent} if there are scalars \hlist ck, at least one of which is nonzero, such that \(\lincomb c\vv k=\ov\)\,.
A set of vectors that is not linearly dependent is called \bfidx{linearly independent} (characterised by \emph{only} the linear combination with \(c_1=c_2=\cdots=c_k=0\) gives the zero vector).
\end{definition}

When reading the terms ``linearly in/dependent'' be very careful: it is all too easy to misread the presence or absence of the crucial ``in''~syllable. 
The presence or absence of the ``in''~syllable makes all the difference between the property and its opposite.


\begin{example} \label{eg:lindep}
Are the following sets of vectors linearly dependent or linearly independent.  Give reasons.
\begin{enumerate}
\item\label{eg:lindepa} \(\{(-1,1,0),\ (1,-2,1),\ (0,1,-1)\}\) 
\begin{solution} 
The set is linearly dependent as the linear combination \((-1,1,0)+(1,-2,1)+(0,1,-1)=(0,0,0)\). 
\end{solution}

\item\label{eg:lindepb} \(\{(2,1),\ (1,2)\}\) 
\begin{solution} 
The set is linearly independent because the linear combination equation \(c_1(2,1)+c_2(1,2)=(0,0)\) is equivalent to the homogeneous matrix-vector system \(\begin{bmatrix} 2&1\\1&2 \end{bmatrix}\cv=\ov\) which has \emph{only} the zero solution \(\cv=\ov\)\,. 
\end{solution}

%\item \(\{(-1,1,0),\ (1,-2,1)\}\)
%\begin{solution} 
%The set is linearly independent: for the linear combination \(c_1(-1,1,0)+c_2(1,-2,1)
%=(-c_1+c_2,c_1-2c_2,c_2)=\ov\) requires the last component to be zero which requires \(c_2=0\);
%then either of the other components requires \(c_1=0\)\,.
%\end{solution}

\item\label{eg:lindepc} \(\{(-2,4,1,-1,0)\}\)
\begin{solution} 
This set of one vector in~\(\RR^5\) is linearly independent as \(c_1(-2,4,1,-1,0)=\ov\) can only be satisfied with \(c_1=0\)\,. 

Indeed, any one non-zero vector~\vv\ in~\(\RR^n\) forms a linearly independent set,~\(\{\vv\}\), for the same reason.
\end{solution}


\item \(\{(2,1),\ (0,0)\}\) 
\begin{solution} 
The set is linearly dependent because the linear combination  \(0(2,1)+c_2(0,0)=(0,0)\) for any non-zero~\(c_2\). 
\end{solution}

\item \(\{\ov,\hlist[2]\vv k\}\)
\begin{solution} 
Any set that includes the zero vector is linearly dependent as
\(c_1\ov+0\vv_2+\cdots+0\vv_k=\ov\) for any non-zero~\(c_1\).
\end{solution}

\item \(\{\ev_1,\ev_2,\ev_3\}\), the set of standard unit vectors in~\(\RR^3\).
\begin{solution} 
This set is linearly independent as
\(c_1\ev_1+c_2\ev_2+c_3\ev_3=(c_1,c_2,c_3)=\ov\)
\emph{only} when all three components are zero, \(c_1=c_2=c_3=0\)\,. 
\end{solution}

\item \(\{(\frac13,\frac23,\frac23),\ (\frac23,\frac13,-\frac23)\}\)
\begin{solution} 
This set is linearly independent.
Seek some linear combination \(c_1(\frac13,\frac23,\frac23) +c_2(\frac23,\frac13,-\frac23) =\ov\)\,.
Take the dot product of both sides of this equation with \((1,2,2)\):
\begin{eqnarray*}
&&c_1\begin{bmatrix}\frac13\\\frac23\\\frac23 \end{bmatrix}\cdot\begin{bmatrix} 1\\2\\2 \end{bmatrix} +c_2\begin{bmatrix} \frac23\\\frac13\\-\frac23 \end{bmatrix}\cdot\begin{bmatrix} 1\\2\\2 \end{bmatrix} =\ov\cdot\begin{bmatrix} 1\\2\\2 \end{bmatrix}
\\\implies&& c_13+c_20=0
\\\implies&& c_1=0\,.
\end{eqnarray*}
Similarly, take the dot product with \((2,1,-2)\): 
\begin{eqnarray*}
&&c_1\begin{bmatrix}\frac13\\\frac23\\\frac23 \end{bmatrix}\cdot\begin{bmatrix} 2\\1\\-2 \end{bmatrix} +c_2\begin{bmatrix} \frac23\\\frac13\\-\frac23 \end{bmatrix}\cdot\begin{bmatrix} 2\\1\\-2 \end{bmatrix} =\ov\cdot\begin{bmatrix} 2\\1\\-2 \end{bmatrix}
\\\implies&& c_10+c_23=0
\\\implies&& c_2=0\,.
\end{eqnarray*}
Hence \(c_1=c_2=0\) and so the vectors are linearly independent.
\end{solution}

\end{enumerate}
These last two cases generalise to the next Theorem~\ref{thm:ortholi} about the linear independence of any orthonormal set of vectors.
\end{example}



\begin{example}[calculus extension] \label{eg:}
In calculus the notion of a function corresponds precisely to the notion of a vector in our linear algebra.  
For the purposes of this example, consider `vector' and `function' to be synonymous, and that `all components' and `all~\(x\)' are synonymous. 
Show that the set \(\{e^x,e^{-x},\cosh x,\sinh x\}\) is linearly dependent.  
What is a subset that is linearly independent?
\begin{solution} 
The definition of the hyperbolic functions, namely that \(\cosh x=(e^x+e^{-x})/2\) and \(\sinh x=(e^x-e^{-x})/2\), immediately give two nontrivial linear combinations that are zero for all~\(x\), namely \(2\cosh x-e^x-e^{-x}=0\) and \(2\sinh x-e^x+e^{-x}=0\) for all~\(x\).
Either one of these implies the set \(\{e^x,e^{-x},\cosh x,\sinh x\}\) is linearly dependent.

Because \(e^x\) and~\(e^{-x}\) are not proportional to each other, there is no linear combination which is zero for all~\(x\), and hence the set \(\{e^x,e^{-x}\}\) is linearly independent (as are any other pairs of the four functions).
\end{solution}
\end{example}






\begin{theorem} \label{thm:ortholi}
Any \idx{orthonormal set} (Definition~\ref{def:orthoset}) of vectors is \idx{linearly independent}.
\end{theorem}
\begin{proof} %Take dot products with the defining equation.
Let \(\{\hlist\vv k\}\) be an orthonormal set of vectors in~\(\RR^n\).
Let's find all possible scalars \hlist ck such that \(\lincomb c\vv k=\ov\)\,.
Taking the dot product of this equation with~\(\vv_1\) implies
\(c_1\vv_1\cdot\vv_1+c_2\vv_2\cdot\vv_1+\cdots+c_k\vv_k\cdot\vv_1=\ov\cdot\vv_1\)\,;
by orthonormality this identity becomes
\(c_11+c_20+\cdots+c_k0=0\)\,; that is, \(c_1=0\)\,.
Similarly, taking the dot product with~\(\vv_2\) implies
\(c_1\vv_1\cdot\vv_2+c_2\vv_2\cdot\vv_2+\cdots+c_k\vv_k\cdot\vv_2=\ov\cdot\vv_2\)\,;
by orthonormality this identity becomes
\(c_10+c_21+\cdots+c_k0=0\)\,; that is, \(c_2=0\)\,.
And so on for all vectors in the set, implying the coefficients \(c_1=c_2=\cdots=c_k=0\) is the only possibility.
By Definition~\ref{def:lindep}, the orthonormal set must be linearly independent.
\end{proof}


In contrast to orthonormal vectors which are always linearly independent, a set of two vectors proportional to each other is always linearly dependent as seen in the following examples.
This linear dependence of proportional vectors then generalises in the next Theorem~\ref{thm:lindeplc}.
\begin{example} \label{eg:}
Show the following sets are linearly dependent.
\begin{enumerate}
\item \(\{(1,2),\ (3,6)\}\)
\begin{solution} 
Since \((3,6)=3(1,2)\) then the linear combination \(1(3,6)-3(1,2)=\ov\) and the set is linearly dependent. 
\end{solution}

\item \(\{(2.2,-2.1,0,1.5),\ (-8.8,8.4,0,-6)\}\)
\begin{solution} 
Since  \((-8.8,8.4,0,-6)=-4(2.2,-2.1,0,1.5)\) then the linear combination 
\begin{equation*}
(-8.8,8.4,0,-6)+4(2.2,-2.1,0,1.5)=\ov\,,
\end{equation*}
and so the set is linearly dependent.
\end{solution}
\end{enumerate}
\end{example}




\begin{theorem} \label{thm:lindeplc} 
A set of vectors \(\{\hlist\vv m\}\) is \idx{linearly dependent} if and only if at least one of the vectors can be expressed as a \idx{linear combination} of the other vectors.
In particular, a set of two vectors \(\{\vv_1,\vv_2\}\) is linearly dependent if and only if one of the vectors is a multiple of the other.
\end{theorem}

\begin{proof} 
Exercise~\ref{ex:lindeplc} establishes the particular case of a set of two vectors.
%Start with the particular case of the set of two vectors \(\{\vv_1,\vv_2\}\).
%Say \(\vv_1\) is a multiple of~\(\vv_2\), \(\vv_1=c\vv_2\)\,, then the linear combination \(\vv_1-c\vv_2=\ov\)\,, and so the set is linearly dependent. 
%Similarly if \(\vv_2\) is a multiple of~\(\vv_1\).
%Conversely, if the set is linearly dependent then \(c_1\vv_1+c_2\vv_2=0\) where one or both of \(c_1,c_2\neq0\): if \(c_1\neq 0\) then rearrange to \(\vv_1=-(c_2/c_1)\vv_2\) and so \(\vv_1\) is a multiple of~\(\vv_2\); and similarly if \(c_2\neq 0\)\,.

In the general case of \(m\)~vectors, first establish that if one of the vectors can be expressed as a {linear combination} of the others, then the set is linearly dependent.
Suppose we have labelled the set of vectors so that it is vector~\(\vv_1\) which is a linear combination of the others; that is, \(\vv_1=\lincomb[2]c\vv m\)\,.
Rearranging, \((-1)\vv_1+\lincomb[2]c\vv m=\ov\)\,; that is, there is a non-trivial (as at least \(c_1=-1\neq0\)) linear combination of the set of vectors which is zero.
Hence the set is linearly dependent.

Second, establish the converse.  
Given the set is linearly dependent, there exist coefficients, not all zero, such that \(\lincomb c\vv m=\ov\)\,.  
Suppose that we have labelled the vectors so that \(c_1\neq 0\)\,.  
Then rearranging the equation gives
\(c_1\vv_1=-c_2\vv_2-c_3\vv_3-\cdots-c_m\vv_m\)\,.
Divide by the non-zero~\(c_1\) to deduce
\(\vv_1=-(c_2/c_1)\vv_2-(c_3/c_1)\vv_3-\cdots-(c_m/c_1)\vv_m\)\,;
that is, \(\vv_1\)~is a linear combination of the other vectors.
\end{proof}


\begin{example} \label{eg:}
Invoke Theorem~\ref{thm:lindeplc} to deduce whether the following sets are linearly independent or linearly dependent.
\begin{enumerate}
\item \(\{(-1,1,0),\ (1,-2,1),\ (0,1,-1)\}\)
\begin{solution} 
Since \((1,-2,1)=-(-1,1,0)-(0,1,-1)\) the set must be linearly dependent.
\end{solution}

\item The set of two vectors shown in the margin.
\marginpar{\begin{tikzpicture}
\begin{axis}[footnotesize,font=\footnotesize
  ,axis lines=none, axis equal image
  ]
\addplot[blue,quiver={u=3,v=2},-stealth,thick] coordinates {(0,0)};
\node[below] at (axis cs:3,2) {$\vv_2$};
\addplot[blue,quiver={u=1,v=2},-stealth,thick] coordinates {(0,0)};
\node[right] at (axis cs:1,2) {$\vv_1$};
\addplot[black,mark=*] coordinates {(0,0)};
\end{axis}
\end{tikzpicture}}
\begin{solution} 
Since they are not proportional to each other, we cannot write either as a multiple of the other, and so the pair are linearly independent. 
\end{solution}

\item The set of two vectors shown in the margin.
\marginpar{\begin{tikzpicture}
\begin{axis}[footnotesize,font=\footnotesize
  ,axis lines=none, axis equal image
  ]
\addplot[blue,quiver={u=3,v=2},-stealth,thick] coordinates {(0,0)};
\node[below] at (axis cs:3,2) {$\vv_2$};
\addplot[blue,quiver={u=-1,v=-0.667},-stealth,thick] coordinates {(0,0)};
\node[above] at (axis cs:-1,-0.667) {$\vv_1$};
\addplot[black,mark=*] coordinates {(0,0)};
\end{axis}
\end{tikzpicture}}
\begin{solution} 
Since they appear proportional to each other, \(\vv_2\approx (-3)\vv_1\)\,,  so the pair appear linearly dependent. 
\end{solution}

\item \(\{(1,3,0,-1),\ (1,0,-4,2),\ (-2,3,0,-3),\ (0,6,-4,-2)\}\)
\begin{solution} 
Notice that the last vector is the sum of the first three, \((0,6,-4,-2)=(1,3,0,-1)+(1,0,-4,2)+(-2,3,0,-3)\), and so the set is linearly dependent. 
\end{solution}

\end{enumerate}
\end{example}




Recall that Theorem~\ref{thm:orthoevec} established that for every two distinct {eigenvalue}s of a symmetric matrix~\(A\), any corresponding two {eigenvector}s are {orthogonal}.
Consequently, for a symmetric~\(A\), a set of eigenvectors from distinct eigenvalues forms an orthogonal set.
The following Theorem~\ref{thm:indepev} generalises this property to non-symmetric matrices using the concept of linear independence.


\begin{theorem} \label{thm:indepev}
For an \(n\times n\) matrix~\(A\), let \hlist\lambda m\ be distinct \idx{eigenvalue}s of~\(A\) with corresponding \idx{eigenvector}s \hlist\vv m.
The set \(\{\hlist \vv m\}\) is \idx{linearly independent}.
\end{theorem}
\begin{proof} 
Use contradiction.
Assume the set \(\{\hlist \vv m\}\) is linearly dependent.
Choose~\(k<m\) such that the set \(\{\hlist\vv k\}\) is linearly independent, whereas the set \(\{\hlist\vv{k+1}\}\) is linearly dependent.
Hence there exists non-trivial coefficients such that 
\begin{equation*}
\lincomb c\vv{k}+c_{k+1}\vv_{k+1}=\ov\,;
\end{equation*}
further, \(c_{k+1}\neq0\) as \(\{\hlist\vv k\}\) is linearly independent.
Pre-multiply the linear combination by matrix~\(A\):
\begin{eqnarray*}&&
\lincomb c{A\vv}k+c_{k+1}A\vv_{k+1}=A\ov
\\\implies&&
c_1\lambda_1\vv_1+c_2\lambda_2\vv_2+\cdots+c_k\lambda_k\vv_k+c_{k+1}\lambda_{k+1}\vv_{k+1}=\ov\,.
\end{eqnarray*}
Now subtract \(\lambda_{k+1}\times\) the original linear combination:
\begin{eqnarray*}&&
\phantom{{}+(}c_1\lambda_1\vv_1+c_2\lambda_2\vv_2+\cdots+c_k\lambda_k\vv_k+c_{k+1}\lambda_{k+1}\vv_{k+1}
\\&&{}
-\left(\lincomb c{\lambda_{k+1}\vv}{k}
+c_{k+1}\lambda_{k+1}\vv_{k+1}\right)=\ov
\\\implies&&
c_1(\lambda_1-\lambda_{k+1})\vv_1+c_2(\lambda_2-\lambda_{k+1})\vv_2+\cdots+c_k(\lambda_k-\lambda_{k+1})\vv_k=\ov
\\\implies&&
\lincomb{c'}\vv k=\ov
\end{eqnarray*}
for coefficients \(c'_j=c_j(\lambda_j-\lambda_{k+1})\).
Since all the eigenvalues are distinct, \(\lambda_j-\lambda_{k+1}\neq0\)\,, and since the coefficients~\(c_j\) are not all zero, hence \(c'_j\)~are not all zero.
Thus we have created a non-trivial linear combination of \hlist \vv k\ which is zero, and so the set \(\{\hlist\vv k\}\) is linearly dependent.
This contradiction of the choice of~\(k\) proves the assumption must be wrong.
Hence the set \(\{\hlist \vv m\}\) is linearly independent, as required.
\end{proof}







\begin{example} \label{eg:indepev}
For each of the following matrices, show the eigenvectors from distinct eigenvalues form linearly independent sets.
\begin{enumerate}
\item  Recall from Example~\ref{eg:faespm} the matrix \(B=\begin{bmatrix}-1&1&-2
\\-1&0&-1
\\0&-3&1 \end{bmatrix}\)
\begin{solution} 
In \script, executing 
\begin{verbatim}
B=[-1 1 -2
 -1 0 -1
 0 -3 1]
[V,D]=eig(B)
\end{verbatim}
\setbox\ajrqrbox\hbox{\qrcode{% eigenvectors
B=[-1 1 -2
 -1 0 -1
 0 -3 1]
[V,D]=eig(B)
}}%
\marginpar{\usebox{\ajrqrbox}}%
gives eigenvectors and corresponding eigenvalues in
\begin{verbatim}
V =
    -0.5774     0.7071    -0.7071
    -0.5774     0.0000     0.0000
    -0.5774    -0.7071     0.7071
D =
         -2          0          0
          0          1          0
          0          0          1
\end{verbatim} 
Recognising \(0.7071=1/\sqrt2\)\,, the last two eigenvectors, \((1/\sqrt2,0,-1/\sqrt2)\) and  \((-1/\sqrt2,0,1/\sqrt2)\), form a linearly dependent set because they are proportional to each other.
This linear dependence does not confound  Theorem~\ref{thm:linhomo} because the corresponding eigenvalues are the same, not distinct, namely \(\lambda=1\)\,.
The theorem only applies to eigenvectors of distinct eigenvalues.

Here the two distinct eigenvalues are \(\lambda=-2\) and \(\lambda=1\)\,.
Recognising \(0.5774=1/\sqrt3\)\,, two corresponding eigenvectors are \((-1/\sqrt3,-1/\sqrt3,-1/\sqrt3)\) and \((1/\sqrt2,0,-1/\sqrt2)\).
Because of the zero component in the second, these cannot be proportional to each other, and so the pair form a linearly independent set.
\end{solution}


\item\label{eg:indepeva} Example~\ref{eg:fiveev} found the eigenvalues and eigenvectors of matrix
\begin{equation*}
A=\begin{bmatrix}0&3&0&0&0
\\1&0&3&0&0
\\0&1&0&3&0
\\0&0&1&0&3
\\0&0&0&1&0\end{bmatrix}
\end{equation*}
In \script\ execute
\begin{verbatim}
A=[0 3 0 0 0
 1 0 3 0 0
 0 1 0 3 0
 0 0 1 0 3
 0 0 0 1 0]
[V,D]=eig(A)
\end{verbatim}
\setbox\ajrqrbox\hbox{\qrcode{% eigenvectors
A=[0 3 0 0 0
 1 0 3 0 0
 0 1 0 3 0
 0 0 1 0 3
 0 0 0 1 0]
[V,D]=eig(A)
svd(V)
}}%
\marginpar{\usebox{\ajrqrbox}}%
to obtain the report \twodp
\begin{verbatim}
V =
   0.62  -0.62   0.94  -0.85  -0.85
   0.62   0.62  -0.00   0.49  -0.49
   0.42  -0.42  -0.31  -0.00   0.00
   0.21   0.21  -0.00  -0.16   0.16
   0.07  -0.07   0.10   0.09   0.09
D =
   3.00      0      0      0      0
      0  -3.00      0      0      0
      0      0  -0.00      0      0
      0      0      0  -1.73      0
      0      0      0      0   1.73
\end{verbatim}
The five eigenvalues are all distinct, so Theorem~\ref{thm:linhomo} asserts a set of corresponding eigenvectors will be linearly independent.
The five columns of~\(V\), call them \hlist\vv5, are a set of corresponding eigenvectors.
To confirm their linear independence let's seek a linear combination being zero, that is, \(\lincomb c\vv5=\ov\)\,.
Written as a matrix-vector system we seek \(\cv=(\hlist c5)\) such that \(V\cv=\ov\)\,.
Because the five singular values of square matrix~\(V\) are all non-zero,\footnote{One could alternatively compute the determinant \(\texttt{det(V)}=0.09090\) and because it is non-zero Theorem~\ref{thm:ftim3} asserts that the equation has only the solution \(\cv=\ov\)\,.} obtained from \verb|svd(V)| as
\begin{verbatim}
ans =
   1.7703
   1.1268
   0.6542
   0.3625
   0.1922
\end{verbatim}
consequently Theorem~\ref{thm:ftim2} asserts \(V\cv=\ov\) has only the zero solution.
Hence, by Definition~\ref{def:lindep} the set of eigenvectors in the columns of~\(V\) are linearly independent. 

\end{enumerate}
\end{example}


This last case of Example~\ref{eg:indepeva} connects the concept of linear (in)dependence to the existence or otherwise of non-zero solutions to a homogeneous system of linear equations, \(V\cv=\ov\)\,.
So does Example~\ref{eg:lindepb}.
The great utility of this connection is that we understand a lot about homogeneous systems of linear equations.
The next Theorem~\ref{thm:linhomo} establishes this connection in general.


\begin{theorem} \label{thm:linhomo} 
Let \hlist\vv m\ be vectors in~\(\RR^n\).  
Let \(n\times m\) matrix \(V=\begin{bmatrix} \vv_1&\vv_2&\cdots&\vv_m \end{bmatrix}\).  
Then \(\{\hlist\vv m\}\) is \idx{linearly dependent} if and only if \(V\cv=\ov\) has a nonzero solution~\cv.
\end{theorem}
\begin{proof} 
Now \(\{\hlist\vv m\}\) is \idx{linearly dependent} if and only if there are scalars, not all zero, such that the equation \(\lincomb c\vv m=\ov\) holds (Definition~\ref{def:lindep}).
Let the vector \(\cv=(\hlist cm)\), then this equation is equivalent to the statement \(V\cv=\ov\)\,.
That is, if and only if \(V\cv=\ov\) has a nonzero solution.
\end{proof}





Recall Theorem~\ref{thm:orthcomp} that in~\(\RR^n\) there can be no more that \(n\)~vectors in an orthogonal set of vectors.
The following theorem is the generalisation: in~\(\RR^n\) there can be no more than \(n\)~vectors in a linearly independent set of vectors.

\begin{theorem} \label{thm:mgtnli} 
Any set of \(m\)~vectors in~\(\RR^n\) is \idx{linearly dependent} when the number of vectors \(m>n\)\,.
\end{theorem}
\begin{proof} 
Form the \(m\)~vectors \(\hlist\vv m\in\RR^n\)\ into the \(n\times m\) matrix \(V=\begin{bmatrix} \vv_1&\vv_2&\cdots&\vv_m \end{bmatrix}\).
Consider the homogeneous system \(V\cv=\ov\)\,: 
%it has nonzero solutions iff the nullspace has nonzero dimension (Theorem~\ref{thm:homosubsp}).
%Recall (Definition~\ref{def:nullity}) that the dimension of the nullspace is the nullity, and the Rank Theorem~\ref{thm:rank} gives
%\begin{eqnarray*}
%\nullity A&=&m-\rank A
%\quad(\text{as \(m\) is the number of columns})
%\\&\geq&m-n
%\quad(\text{as }\rank A\leq \min(m,n)=n\text{ here})
%\\&>&0\quad(\text{again as }m>n\text{ here}).
%\end{eqnarray*}
as \(m>n\)\,, Theorem~\ref{thm:feweqns} (with the meaning of \(m\) and~\(n\) swapped) asserts \(V\cv=\ov\) has infinitely many solutions.
Thus \(V\cv=\ov\) has nonzero solutions, so Theorem~\ref{thm:linhomo} implies the set of eigenvectors \(\{\hlist\vv m\}\) is linearly dependent.
\end{proof}



\begin{example} \label{eg:}
%Use Theorem~\ref{thm:linhomo} and Theorem~\ref{thm:mgtnli}
Determine if the following sets of vectors are linearly dependent or independent.  Give reasons.

\begin{enumerate}
\item \(\{(-1,-2), (-1,4), (0,5), (2,3)\}\)
\begin{solution} 
As there are four vectors in~\(\RR^2\) so Theorem~\ref{thm:mgtnli} asserts the set is linearly dependent.
\end{solution}

\item \(\{(-6,-4,-1,-2),\ (2,0,1,-2),\ (2,-1,-1,1)\}\)
\begin{solution} 
In \script\ form the matrix with these vectors as columns
\begin{verbatim}
V=[-6   2   2
  -4   0  -1
  -1   1  -1
  -2  -2   1]
svd(V)
\end{verbatim}
\setbox\ajrqrbox\hbox{\qrcode{% lin indep
V=[-6   2   2
  -4   0  -1
  -1   1  -1
  -2  -2   1]
svd(V)
}}%
\marginpar{\usebox{\ajrqrbox}}%
and find the three singular values are all non-zero (namely \(7.7568\), \(2.7474\), and~\(2.2988\)).
Hence there are no free variables when solving \(V\cv=\ov\) (Procedure~\ref{pro:gensol}), and consequently there is only the unique solution \(\cv=\ov\)\,.
By Theorem~\ref{thm:linhomo}, the set of vectors is linearly independent.
\end{solution}

\item \(\{(-1,-2,2,-1),\ (1,3,1,-1),\ (-2,-4,4,-2)\}\)
\begin{solution} 
By inspection, the third vector is twice the first.
Hence the linear combination \(2(-1,-2,2,-1)+0(1,3,1,-1)-(-2,-4,4,-2)=\ov\) and so the set of vectors is linearly dependent. 
\end{solution}

\item \(\{(3,3,-1,-1),\ (0,-3,-1,-7),\ (1,2,0,2)\}\)
\begin{solution} 
In \script\ form the matrix with these vectors as columns
\begin{verbatim}
V=[3   0   1
   3  -3   2
  -1  -1   0
  -1  -7   2]
svd(V)
\end{verbatim}
\setbox\ajrqrbox\hbox{\qrcode{% lin indep
V=[3   0   1
   3  -3   2
  -1  -1   0
  -1  -7   2]
svd(V)
}}%
\marginpar{\usebox{\ajrqrbox}}%
and find the three singular values are \(8.1393\), \(4.6638\), and~\(0.0000\).
The zero singular value implies there is a free variables when solving \(V\cv=\ov\) (Procedure~\ref{pro:gensol}), and consequently there are infinitely many non-zero~\cv\ that solve \(V\cv=\ov\).
By Theorem~\ref{thm:linhomo}, the set of vectors is linearly dependent.
\end{solution}

\item \(\{(10,3,3,1)\), \(( 2,-3,0,-1)\), \(( 1,-1,2,-1)\), \(( 2,-1,-3,0)\), \((-2,0,2,-1)\}\)
\begin{solution} 
As there are five vectors in~\(\RR^4\) so Theorem~\ref{thm:mgtnli} asserts the set is linearly dependent.
\end{solution}



\item \(\{(-0.4,-1.8,-0.2, 0.7,-0.2)\), \((-1.1, 2.8, 2.7,-3.0,-2.6)\), \((-2.3,-2.3, 4.1, 3.4,-1.6)\), \((-2.6,-5.3,-3.3,-1.3,-4.1)\), \(( 1.4, 5.2,-6.9,-0.7, 0.6)\}\)
\begin{solution} 
In \script\ form the matrix~\(V\) with these vectors as columns
\begin{verbatim}
V=[-0.4 -1.1 -2.3 -2.6 1.4
  -1.8  2.8 -2.3 -5.3  5.2
  -0.2  2.7  4.1 -3.3 -6.9
   0.7 -3.0  3.4 -1.3 -0.7
  -0.2 -2.6 -1.6 -4.1  0.6]
svd(V)
\end{verbatim}
\setbox\ajrqrbox\hbox{\qrcode{% lin indep
V=[-0.4 -1.1 -2.3 -2.6 1.4
  -1.8  2.8 -2.3 -5.3  5.2
  -0.2  2.7  4.1 -3.3 -6.9
   0.7 -3.0  3.4 -1.3 -0.7
  -0.2 -2.6 -1.6 -4.1  0.6]
svd(V)
}}%
\marginpar{\usebox{\ajrqrbox}}%
and find the five singular values are \(10.6978\), \(8.0250\), \(5.5920\), \(3.0277\) and~\(0.0024\).
As the singular values are all non-zero, the homogeneous system \(V\cv=\ov\) has the unique solution \(\cv=\ov\) (Procedure~\ref{pro:gensol}), and hence the set of five vectors are linearly independent.

However, the answer depends upon the context.  
In the strict mathematical context the vectors are unequivocally linearly independent.
But in the context of practical problems, where errors in matrix entries are likely, there are `shades of grey' . 
Here, one of the singular values is quite small, namely~\(0.0024\).
If the context informs us that the entries in the matrix had errors of say~\(0.01\), then this singular value is effectively zero (Section~\ref{sec:rle}).
In the context of such errors, this set of five vectors would be effectively linearly dependent.
\end{solution}


\end{enumerate}
%for i=1:9999, a=round(randn(3,4)*3);if min(svd(a))<1e-7,null(a'),break,end,end
\end{example}













\subsection{Form a basis for subspaces}


Recall from Sections~\ref{sec:lcss} and~\ref{sec:sbd} the definition of \idx{subspace}s and the span: namely that a subspace is a set of vectors closed under addition and scalar multiplication; and a span is a set of vectors whose linear combinations give all vectors in a given subspace.
Also, Definition~\ref{def:orthobasis} defined an ``\idx{orthonormal basis}'' for a subspace to be a set of orthonormal vectors that span a subspace.
This section generalises orthonormal basis by relaxing the requirement of orthonormality.


\begin{definition} \label{def:basis} 
A \bfidx{basis} for a \idx{subspace}~\WW\ of~\(\RR^n\) is a set of  vectors that both \idx{span}~\WW\ and is \idx{linearly independent}.
\end{definition}

\begin{example} \label{eg:}
\begin{enumerate}
\item Recall Examples~\ref{eg:lindepb} and~\ref{eg:2dnoc} showed that the two vectors \((2,1)\) and \((1,2)\) are linearly independent and span~\(\RR^2\). 
Hence the set \(\{(2,1),\, (1,2)\}\) is a basis of~\(\RR^2\).

\item Recall that Example~\ref{eg:lindepa} showed the set \(\{(-1,1,0)\), \((1,-2,1)\), \((0,1,-1)\}\) is linearly dependent so it cannot be a basis.

However, remove one vector, such as the middle one, and consider the set \(\{(-1,1,0)\), \((0,1,-1)\}\).
As the two vectors are not proportional to each other, this set is linearly independent (Theorem~\ref{thm:lindeplc}).
Also, the plane \(x+y+z=0\) is a subspace, say~\WW.
It is characterised by \(y=-x-z\)\,.
So every vector in~\WW\ can be written as \((x,-x-z,z)=(x,-x,0)+(0,-z,z)=(-x)(-1,1,0)+(-z)(0,1,-1)\). 
That is, \(\Span\{(-1,1,0)\), \((0,1,-1)\}=\WW\).
Hence \(\{(-1,1,0)\), \((0,1,-1)\}\) is a basis for the plane~\WW.

\item Find a basis for the line given parametrically as \(x=2.1t\)\,, \(y=1.3t\) and \(z=-1.1t\) (shown below in stereo).
\begin{center}
\qview{30}{34} {\begin{tikzpicture}
\begin{axis}[small,font=\footnotesize, view={\q}{30}
  ,xlabel={$x$}, ylabel={$y$}, zlabel={$z$},label shift={-2ex}
  ,domain=-2:2,axis equal image
  ] 
\addplot3[blue,samples=3,no marks]  ({2.1*x},{1.3*x},{-1.1*x});
\addplot3[blue,thick,quiver={u=2.1,v=1.3,w=-1.1},-stealth,mark=*] coordinates {(0,0,0)};
\node[below] at (axis cs:2.1,1.3,-1.1) {$(2.1,1.3,-1.1)$};
\end{axis}
\end{tikzpicture}}
\end{center}
\begin{solution} 
The vectors in the line may be written as \(\xv=(x,y,z) =(2.1t,1.3t,-1.1t) =(2.1,1.3,-1.1)t\)\,.
Since the parameter~\(t\) may vary over all values, vectors in the line form \(\Span\{(2.1,1.3,-1.1)\}\). 
Since \(\{(2.1,1.3,-1.1)\}\) is a linearly independent set of vectors (Example~\ref{eg:lindepc}), it thus forms a basis for the vectors in the given line.
\end{solution}


\item Find a basis for the line given parametrically as \(x=5.7t-0.6\) and \(y=6.8t+2.4\)\,.
\begin{solution} 
The vectors in the line may be written as \(\xv=(5.7t-0.6,6.8t+2.4)\)\,.
\marginpar{\begin{tikzpicture}
\begin{axis}[footnotesize,font=\footnotesize
  ,axis lines=middle, axis equal image
  ,xlabel={$x$}, ylabel={$y$}
  ,thick,no marks,samples=3 ,domain=-1:1 ] 
\addplot  ({5.7*x-0.6},{6.8*x+2.4});
\end{axis}
\end{tikzpicture}}
But this does not form a subspace as it does not include the zero vector~\ov\ (as illustrated in the margin): 
the \(x\)-component is zero for positive~\(t\) whereas the \(y\)-component is zero for negative~\(t\) so they are never zero for the same value of parameter~\(t\).
Since this line is not a subspace, it cannot have a basis.
\end{solution}




\item Find a basis for the plane \(3x-2y+z=0\)\,.
\begin{solution} 
Writing the equation of the plane as \(z=-3x+2y\) we then write the plane parametrically (section~\ref{sec:nvep}) as the vectors \(\xv=(x,y,-3x+2y) =(x,0,-3x)+(0,y,2y) =x(1,0,-3) +y(0,1,2)\).
Since \(x\) and~\(y\) may vary over all values, the plane is the subspace \(\Span\{(1,0,-3),\,(0,1,2)\}\) (as illustrated below in stereo).
Since \((1,0,-3)\) and \((0,1,2)\) are not proportional to each other, they form a linearly independent set.
Hence \(\{(1,0,-3),\,(0,1,2)\}\) is a basis for the plane.
\begin{center}
\qview{30}{34} {\begin{tikzpicture}
\begin{axis}[footnotesize,font=\footnotesize, view={\q}{30}
  ,xlabel={$x$}, ylabel={$y$}, zlabel={$z$},label shift={-2ex}
  ,no marks ,domain=-2:2 ] 
\addplot3[surf,blue,samples=3,opacity=0.3]  {-3*x+2*y};
\addplot3[blue,thick,quiver={u=1,v=0,w=-3},-stealth] coordinates {(0,0,0)};
\node[below] at (axis cs:1,0,-3) {$(1,0,-3)$};
\addplot3[blue,thick,quiver={u=0,v=1,w=2},-stealth] coordinates {(0,0,0)};
\node[above] at (axis cs:0,1,2) {$(0,1,2)$};
\end{axis}
\end{tikzpicture}}
\end{center}
\end{solution}




\item %keep this as the last case of this example
Prove that every orthonormal basis of a subspace~\WW\ is also a basis of~\WW.
\begin{solution} 
Theorem~\ref{thm:ortholi} establishes that any orthonormal basis is linearly independent.
By Definition~\ref{def:orthobasis}, an orthonormal basis of~\WW\ spans~\WW.
Hence an orthonormal basis of~\WW\ is also a basis of~\WW.
\end{solution}
\end{enumerate}
\end{example}


Recall that Theorem~\ref{thm:sameD} establishes that an \idx{orthonormal basis} of a given subspace always has the same number of vectors.
The following theorem establishes the same is true for general bases.
The proof is direct generalisation of that for Theorem~\ref{thm:sameD}.

\begin{theorem} \label{thm:sameDii} 
Any two bases for a given \idx{subspace} have the same number of vectors.
\end{theorem}
\begin{proof} 
Let \(\cU=\{\hlist\uv r\}\) and \(\cV=\{\hlist\vv s\}\) be any two  bases for a subspace in~\(\RR^n\).
Prove the number of vectors \(r=s\) by contradiction.
In the first case, assume \(r<s\)\,.
Since \cU\ is a basis for the subspace every vector in the set~\cV\ can be written as a linear combination of vectors in~\cU\ with some coefficients~\(a_{ij}\):
\begin{eqnarray*}
  &&\vv_1=\uv_1a_{11}+\uv_2a_{21}+\cdots+\uv_ra_{r1}\,,
\\&&\vv_2=\uv_1a_{12}+\uv_2a_{22}+\cdots+\uv_ra_{r2}\,,
\\&&\quad\vdots
\\&&\vv_s=\uv_1a_{1s}+\uv_2a_{2s}+\cdots+\uv_ra_{rs}\,.
\end{eqnarray*}
Write each of these, such as the first one, in the form
\begin{equation*}
\vv_1=\begin{bmatrix} \uv_1&\uv_2&\cdots&\uv_r \end{bmatrix}
\begin{bmatrix} a_{11}\\a_{21}\\\vdots\\a_{r1} \end{bmatrix}
=U\av_1\,,
\end{equation*}
where \(n\times r\) matrix \(U=\begin{bmatrix} \uv_1&\uv_2&\cdots&\uv_r \end{bmatrix}\).
Similarly for the other equations \(\vv_2=\cdots=U\av_2\) through to \(\vv_s=\cdots=U\av_s\)\,.
Then the \(n\times s\) matrix
\begin{eqnarray*}
V&=&\begin{bmatrix} \vv_1&\vv_2&\cdots&\vv_s \end{bmatrix}
\\&=&\begin{bmatrix} U\av_1& U\av_2&\cdots&U\av_s\end{bmatrix}
\\&=&U\begin{bmatrix} \av_1& \av_2&\cdots&\av_s\end{bmatrix}
=UA
\end{eqnarray*}
for the \(r\times s\) matrix~\(A=\begin{bmatrix} \av_1& \av_2&\cdots&\av_s \end{bmatrix}\).
By assumption \(r<s\) and so Theorem~\ref{thm:feweqns} assures us that the homogeneous system \(A\xv=\ov\) has infinitely many solutions, choose any non-trivial solution \(\xv\neq\ov\)\,.
Consider 
\begin{eqnarray*}
V\xv&=& UA\xv\quad(\text{from above})
\\&=&U\ov\quad(\text{since }A\xv=\ov)
\\&=&\ov\,.
\end{eqnarray*}
The identity \(V\xv=\ov\) implies there is a linear combination of the columns \hlist\vv s\ of~\(V\) which gives zero, hence the set~\cV\ is linearly dependent (Theorem~\ref{thm:linhomo}).  
But this is a contradiction, so we cannot have \(r<s\)\,.

Second, a corresponding argument establishes we cannot have \(s<r\)\,.
Hence \(s=r\)\,: all bases of a given subspace must have the same number of vectors.
\end{proof}



\begin{example} \label{eg:samedi}
Consider the plane \(x+y+z=0\) in~\(\RR^3\).
Each of the following are a basis for the plane:
\begin{itemize}
\item \(\{(-1,1,0),\ (1,-2,1)\}\);
\item \(\{(1,-2,1),\ (0,1,-1)\}\); 
\item \(\{(0,1,-1),\ (-1,1,0)\}\).
\end{itemize}
The reasons are that all three vectors involved are in the plane, and that each pair are linearly independent (as, in each pair, one is not proportional to the other).
However, although each of the three vectors in the set \(\{(-1,1,0),\ (1,-2,1),\ (0,1,-1)\}\) is in the plane, this set is not a basis because it is not linearly independent (Example~\ref{eg:lindepa}).
Also, each individual vector, say \((-1,1,0)\), cannot form a basis for the plane because the span of one vector, such as \(\Span\{(-1,1,0)\}\), is a line not the whole plane.

The orthonormal basis \(\big\{(1,0,-1)/\sqrt2,\ (1,-2,1)/\sqrt6\big\}\) is another basis for the plane: both vectors satisfy \(x+y+z=0\) and are orthogonal and so linearly independent (Theorem~\ref{thm:ortholi}).  
All these bases possess two vectors.
\end{example}

That all bases for a given subspace, including orthonormal bases, have the same number of vectors leads to the following theorem about the dimensionality.

\begin{theorem} \label{thm:dimii} 
Let \WW\ be a \idx{subspace} of~\(\RR^n\).  
The \bfidx{dimension} of~\WW, denoted~\(\dim\WW\), is the number of vectors in any \idx{basis} for~\WW. 
\end{theorem}

\begin{proof} 
Recall Definition~\ref{def:dim} defined \(\dim\WW\) to be the number of vectors in any orthonormal basis for~\WW.
Theorem~\ref{thm:ortholi} certifies that all orthonormal bases are also bases (Definition~\ref{def:basis}), so Theorem~\ref{thm:sameDii} implies every basis of~\WW\ has \(\dim\WW\) vectors.
\end{proof}





\begin{procedure}[basis for a span] \label{pro:bfs}
To find a \idx{basis} for the \idx{subspace} \(\AA=\Span\{\hlist\av n\}\) given $\{\hlist\av n\}$ is a set of $n$~vectors in~\(\RR^m\).
Recall Procedure~\ref{pro:ospan} underpins finding an \idx{orthonormal basis} by the following.
\begin{enumerate}
\item Form matrix $A:= \begin{bmatrix} \av_1 & \av_2& \cdots&\av_n \end{bmatrix}$. 
\item Factorise~\(A\) into its \svd, $A=\usv$\,, and let \(r=\rank A\) be the number of nonzero \idx{singular value}s (or effectively nonzero when the matrix has experimental errors, Section~\ref{sec:rle}).
\item The set \(\{\hlist\uv r\}\)  (where \(\uv_j\)~denotes the columns of~$U$) is a \idx{basis}, specifically an \idx{orthonormal basis}, for the \(r\)D subspace~\AA.
\end{enumerate}
Alternatively, if \(r=n\)\,, then the set \(\{\hlist\av n\}\) is \idx{linearly independent} and span the subspace~\AA, and so is also a \idx{basis} for the \(n\)D subspace~\AA.
\end{procedure}



\begin{example} \label{eg:}
Apply Procedure~\ref{pro:bfs} to find a basis for the following sets.
\begin{enumerate}
\item Recall Example~\ref{eg:samedi} identified that any pair of vectors in the set \(\{(-1,1,0)\), \((1,-2,1)\), \((0,1,-1)\}\) forms a basis for the plane that they span.  
Find another basis for the plane.
\begin{solution} 
In \script\ form the matrix with these vectors as columns:
\begin{verbatim}
A=[-1 1 0
   1 -2 1
   0 1 -1]
[U,S,V]=svd(A)
\end{verbatim}
\setbox\ajrqrbox\hbox{\qrcode{% basis
A=[-1 1 0
   1 -2 1
   0 1 -1]
[U,S,V]=svd(A)
}}%
\marginpar{\usebox{\ajrqrbox}}%
Then the \svd\  obtains \twodp
\begin{verbatim}
U =
  -0.41  -0.71   0.58
   0.82   0.00   0.58
  -0.41   0.71   0.58
S =
   3.00      0      0
      0   1.00      0
      0      0   0.00
V = ...
\end{verbatim}
The two non-zero singular values determine \(\rank A=2\) and hence the first two columns of~\verb|U| form an (orthonormal) basis for \(\Span\{(-1,1,0)\), \((1,-2,1)\), \((0,1,-1)\}\).
That is, \(\{0.41(-1,2,-1),\) \(0.71(-1,0,1)\}\) is an (orthonormal) basis.
\end{solution}

\item The span of the three vectors
\begin{equation*}
(-2,0,-4,1,1),\ 
(7,1,2,-1,-5),\ 
(-5,-1,2,3,-2).
\end{equation*}
\begin{solution} 
In \script\ it is often easiest to enter these vectors as rows, and then transpose with the dash operator to form the matrix with these as columns:
\begin{verbatim}
A=[ -2 0 -4 1 1
 7 1 2 -1 -5
 -5 -1 2 3 -2]'
[U,S,V]=svd(A)
\end{verbatim}
\setbox\ajrqrbox\hbox{\qrcode{% basis
A=[ -2 0 -4 1 1
 7 1 2 -1 -5
 -5 -1 2 3 -2]'
[U,S,V]=svd(A)
}}%
\marginpar{\usebox{\ajrqrbox}}%
Then the \svd\ obtains \twodp
\begin{verbatim}
U =
   0.86  -0.32  -0.02   0.40   0.02
   0.12  -0.11  -0.06  -0.40   0.90
   0.22   0.65   0.72   0.07   0.13
  -0.23   0.35  -0.38   0.73   0.37
  -0.38  -0.59   0.58   0.37   0.19
S =
   10.07       0       0
       0    5.87       0
       0       0    3.01
       0       0       0
       0       0       0
V = ...
\end{verbatim}
The three non-zero singular values determine  \(\rank A=3\) and so the original three vectors are linearly independent.
Consequently the original three vectors form a basis for their span.

If you prefer an orthonormal basis, then use the first three columns of~\verb|U|. 
\end{solution}



\item The span of the four vectors
\((1,0,3,-4,0)\), 
\((-1,-1,1,4,2)\), 
\((-3,2,2,2,1)\), 
\((3,-3,2,-2,1)\).
\begin{solution} 
In \script, enter these vectors as rows, and then transpose with the dash operator to form the matrix with these as columns:
\begin{verbatim}
A=[1 0 3 -4 0
 -1 -1 1 4 2
 -3 2 2 2 1
 3 -3 2 -2 1]'
[U,S,V]=svd(A)
\end{verbatim}
\setbox\ajrqrbox\hbox{\qrcode{% basis
A=[1 0 3 -4 0
 -1 -1 1 4 2
 -3 2 2 2 1
 3 -3 2 -2 1]'
[U,S,V]=svd(A)
}}%
\marginpar{\usebox{\ajrqrbox}}%
Then the \svd\ obtains \twodp
\begin{verbatim}
U =
  -0.52   0.11   0.46  -0.71   0.02
   0.28  -0.44  -0.55  -0.61   0.24
  -0.19   0.66  -0.60  -0.17  -0.37
   0.78   0.32   0.36  -0.30  -0.27
   0.10   0.51  -0.00   0.03   0.86
S =
   7.64      0      0      0
      0   4.59      0      0
      0      0   4.30      0
      0      0      0   0.00
      0      0      0      0
V = ...
\end{verbatim}
The three non-zero singular values determine  \(\rank A=3\).
Consequently, the first three columns of~\verb|U| form an orthonormal basis for the span. 

Alternatively, you might notice that the sum of the first two vectors is the sum of the last two vectors.
Consequently, given the rank is three, we obtain three linearly independent vectors by omitting any one.
That is, any three of the given vectors form a basis for the span of the four.
\end{solution}

\end{enumerate}
\end{example}


The procedure is different if the subspace of interest is defined by a system of equations instead of the span of some vectors.

\begin{example} \label{eg:bas2sys}
Find a basis for the solutions of the system in~\(\RR^3\) of \(3x+y=0\) and \(3x+2y+3z=0\)\,.
\begin{solution} 
By hand manipulation, the first equation gives \(y=-3x\)\,; which when substituted into the second gives \(3x-6x+3z=0\)\,, namely \(z=x\)\,.  
That is, all solutions are of the form \((x,-3x,x)\), namely \(\Span\{(1,-3,1)\}\).
Thus a basis for the subspace of solutions is~\(\{(1,-3,1)\}\).
(Infinitely many other bases are possible answers.)
\end{solution}
\end{example}




\begin{example} \label{eg:}
Find a basis for the solutions of \(-2x-y+3z=0\) in~\(\RR^3\).
\begin{solution} 
Rearrange the equation so that one variable is a function of the others, say \(y=-2x+3z\).
Then the vector form of of solutions are \((x,y,z)=(x,-2x+3z,z)=(1,-2,0)x+(0,3,1)z\) in terms of \idx{free variable}s~\(x\) and~\(z\).
Since \((1,-2,0)\) and~\((0,3,1)\) are not proportional to each other, they are linearly independent, and so a basis for the solutions is \(\{(1,-2,0),(0,3,1)\}\).
(Infinitely many other bases are possible answers.)
\end{solution}
\end{example}





\begin{procedure}[basis from equations] \label{pro:bfe}
Suppose we seek a \idx{basis} for a \idx{subspace}~\WW\ defined as the solutions of a system of equations.
\begin{enumerate}
\item Rewrite the system of equations as the \idx{homogeneous} system \(A\xv=\ov\)\,, hence the subspace~\WW\ is the \idx{nullspace} of \(m\times n\) matrix~\(A\).
\item  Adapting Procedure~\ref{pro:gensol} for the specific case of homogeneous systems, first find the \svd\ factorisation \(A=\usv\) and let \(r=\rank A\) be the number of nonzero \idx{singular value}s (or effectively nonzero when the matrix has experimental errors, Section~\ref{sec:rle}).
\item Then the general solution of \(S\yv=\zv=\ov\) is \(\yv=(0,\ldots,0,y_{r+1},\ldots,y_n)\).
Consequently, all possible solutions \(\xv=V\yv\) are spanned by the last \(n-r\) columns of~\(V\), which thus form an \idx{orthonormal basis} for the subspace~\WW.
\end{enumerate}
\end{procedure}


\begin{example} \label{eg:}
Find a \idx{basis} for all solutions to each of the following systems of equations.
\begin{enumerate}
\item \(3x+y=0\) and \(3x+2y+3z=0\) from Example~\ref{eg:bas2sys}.
\begin{solution} 
Form matrix \(A=\begin{bmatrix} 3&1&0\\3&2&3 \end{bmatrix}\) and compute an \svd\ with \verb|[U,S,V]=svd(A)| to obtain \twodp
\begin{verbatim}
U = ...
S =
   5.34      0      0
      0   1.86      0
V =
   0.77   0.56   0.30
   0.42  -0.09  -0.90
   0.48  -0.82   0.30
\end{verbatim}
The two non-zero singular values determine \(\rank A=2\)\,.
Hence the solutions of the system are spanned by the last one column of~\(V\).  
That is, a basis for the solutions is \(\{(0.3,-0.9,0.3)\}\).
\end{solution}

\item \(7x=6y+z+3\) and \(4x+9y+2z+2=0\)\,.
\begin{solution} 
This system is not homogeneous (due to the constant terms, Definition~\ref{def:homosys}), therefore \(\xv=\ov\) is not a solution. Consequently, the solutions of the system cannot form a subspace (Definition~\ref{def:subspace}). 
Thus the concept of a basis does not apply (Definition~\ref{def:basis}). 
\end{solution}


\item \(w+x=z\)\,,
\(3w=x+y+5z\)\,,
\(4x+y+2z=0\)\,.
\begin{solution} 
Rearrange to the matrix-vector system \(A\xv=\ov\) for vector \(\xv=(w,x,y,z)\in\RR^4\) and matrix
\begin{verbatim}
A=[1  1  0 -1
   3 -1 -1 -5
   0  4  1  2]
\end{verbatim}
\setbox\ajrqrbox\hbox{\qrcode{% basis
A=[1  1  0 -1
   3 -1 -1 -5
   0  4  1  2]
[U,S,V]=svd(A)
}}%
\marginpar{\usebox{\ajrqrbox}}%
Enter into \script\ as above and then find an \svd\ with \verb|[U,S,V]=svd(A)| to obtain \twodp
\begin{verbatim}
U = ...
S =
   6.77      0      0      0
      0   3.76      0      0
      0      0   0.00      0
V =
  -0.40   0.45   0.09   0.80
   0.41   0.86  -0.19  -0.25
   0.20   0.10   0.97  -0.07
   0.80  -0.24  -0.10   0.54
\end{verbatim}
There are two non-zero singular values, so \(\rank A=2\)\,.
There are thus \(4-2=2\) free variables in solving \(A\xv=\ov\) leading to a 2D subspace with orthonormal basis of the last two columns of~\(V\).
That is, an orthonormal basis for the subspace of all solutions is \(\{(0.09,-0.19,0.97,-0.10)\), \((0.80,-0.25,-0.07,0.54)\}\). 
\end{solution}

\end{enumerate}
%for i=1:9999,A=round(randn(3,4)*4); if min(svd(A))<1e-7, A=A, break, end, end
\end{example}







Recall this Section~\ref{sec:lisb} started by discussing the need to have a unique representation of solutions to problems.
If we do not have uniqueness, then the ambiguity in algebraic representation ruins basic algebra.
The following theorem assures us that the linear independence of a basis ensures the unique representation that we need.
In essence it says that any basis, whether orthogonal or not, can be used to form a coordinate system.




\begin{example}[a tale of two coordinate systems] \label{eg:}
In the margin are plotted three vectors and the origin.
\marginpar{\begin{tikzpicture}
\newcommand{\ppoint}[3]{
    \pgfmathparse{#1*3+#2*1}\let\h\pgfmathresult
    \pgfmathparse{#1*1+#2*2}\let\v\pgfmathresult
    \addplot[blue,mark=*,thick,quiver={u=\h,v=\v},-stealth] coordinates {(0,0)};
    \edef\tempe{
    \noexpand\node[right] at (axis cs:\h,\v) {$#3$};
    }\tempe
    }
\begin{axis}[footnotesize,font=\footnotesize
  , axis lines=none%,xlabel={$x$},ylabel={$y$}
  , axis equal image
  , view={0}{90}
  ,xmax=8.5,ymax=5.5,xmin=-6.5,ymin=-5.5
  ]
\ppoint21{a}
\ppoint1{-3}{b}
\ppoint{-2}3{c}
\ppoint00{}
\end{axis}
\end{tikzpicture}}%
Take the view that these are fixed physically meaningful vectors: the issue of this example is how do we code such vectors in mathematics.

\begin{tabular}{@{}cc@{}}
\begin{tikzpicture}
\newcommand{\ppoint}[3]{
    \pgfmathparse{#1*3+#2*1}\let\h\pgfmathresult
    \pgfmathparse{#1*1+#2*2}\let\v\pgfmathresult
    \addplot[blue,mark=*,only marks] coordinates {(\h,\v)};
    \edef\tempe{%
    \noexpand\node[left] at (axis cs:\h,\v) {$(\h,\v)$};
    \noexpand\node[right] at (axis cs:\h,\v) {$#3$};
    }\tempe
    }
\begin{axis}[small,font=\footnotesize
  , axis lines=middle,xlabel={$x$},ylabel={$y$},grid
  , axis equal image
  , view={0}{90}
  ,xmax=7.9,ymax=5.5,xmin=-7.1,ymin=-5.5
  ]
\ppoint21{a}
\ppoint1{-3}{b}
\ppoint{-2}3{c}
\ppoint00{}
\end{axis}
\end{tikzpicture}
&\parbox[b]{0.4\linewidth}{
In the standard orthogonal coordinate system these three vectors and the origin have coordinates as plotted left by their end-points.
We write \(\av=(7,4)\), \(\bv=(0,-5)\) and \(\cv=(-3,4)\).}
\end{tabular}


\begin{tabular}{@{}cc@{}}
\begin{tikzpicture}
\newcommand{\ppoint}[3]{
    \pgfmathparse{#1*3+#2*1}\let\h\pgfmathresult
    \pgfmathparse{#1*1+#2*2}\let\v\pgfmathresult
%    \addplot[blue,mark=*,thick,quiver={u=\h,v=\v},-stealth] coordinates {(0,0)};
    \addplot[blue,mark=*,only marks] coordinates {(\h,\v)};
    \edef\tempe{%
    \noexpand\node[left] at (axis cs:\h,\v) {$(#1,#2)$};
%    \node[left] at (axis cs:\h,\v) {$(\h,\v)$};
    \noexpand\node[right] at (axis cs:\h,\v) {$#3$};
    }\tempe
    }
\begin{axis}[small,font=\footnotesize
%  , axis lines=middle,xlabel={$x$},ylabel={$y$},grid
  , axis lines=none
  , axis equal image
  , view={0}{90}
  ,xmax=7.9,ymax=5.5,xmin=-7.1,ymin=-5.5
  ]
\addplot3[mesh,red,samples=9,domain=-4:4,dotted] (3*x+y,x+2*y,0);
\addplot[red,quiver={u=3,v=1},-stealth,thick] coordinates {(0,0)};
\node[right] at (axis cs:3,1) {$\vv_1$};
\addplot[red,quiver={u=1,v=2},-stealth,thick] coordinates {(0,0)};
\node[above] at (axis cs:1,2) {$\vv_2$};
\ppoint21{a}
\ppoint1{-3}{b}
\ppoint{-2}3{c}
\ppoint00{}
\end{axis}
\end{tikzpicture}
&\parbox[b]{0.4\linewidth}{
Now use the (red) basis \(\cB=\{\vv_1,\vv_2\}\) to form a non-orthogonal coordinate system (represented by the dotted grid).
Then in this system the three vectors have coordinates \(\av=(2,1)\), \(\bv=(1,-3)\) and \(\cv=(-2,3)\).}
\end{tabular}

But we cannot say both \(\av=(7,4)\) and \(\av=(2,1)\): it appears nonsense.
The reason for the different numbers representing the one vector~\av\ is that the underlying coordinate systems are different.
For example, we can say both \(\av=7\ev_1+4\ev_2\) and \(\av=2\vv_1+\vv_2\) without any apparent contradiction: these statements recognise the underlying standard unit vectors in the first expression and the underlying non-orthogonal basis vectors in the second.

Consequently we invent a new better notation.
We write \([\av]_\cB=(2,1)\) to represent that the coordinates of vector~\av\ in the basis~\cB\ are~\((2,1)\).
Correspondingly, letting \(\cE=\{\ev_1,\ev_2\}\) denote the basis of the \idx{standard unit vector}s, we write \([\av]_\cE=(7,4)\) to represent that the coordinates of vector~\av\ in the \idx{standard basis}~\cE\ are~\((7,4)\).
Similarly, \([\bv]_\cE=(0,-5)\) and \([\bv]_\cB=(1,-3)\);
and \([\cv]_\cE=(-3,4)\) and \([\cv]_\cB=(-2,3)\).

The endemic practice of just writing \(\av=(2,1)\), \(\bv=(1,-3)\) and \(\cv=(-2,3)\) is rationalised in this new notation by the convention that if no basis is specified, then the \idx{standard basis}~\cE\ is assumed.
\footnote{Given that the numerical representation of a vector changes with the coordinate basis, some of you will wonder whether the same thing happens for matrices.  The answer is yes. }
\end{example}



\begin{theorem} \label{thm:ssbc} 
Let \WW\ be a \idx{subspace} of~\(\RR^n\)  and let \(\cB=\{\hlist\vv k\}\) be a \idx{basis} for~\WW.  
There is exactly one way to write each and every vector \(\wv\in\WW\) as a \idx{linear combination} of the \idx{basis} vectors: \(\wv=\lincomb c\vv k\)\,.
Then \(\hlist ck\) are called the \textbf{\bfidx{coordinates} of~\wv\ with respect to~\cB}, and the column vector \([\wv]_{\cB}=(\hlist ck)\) is called the \textbf{\bfidx{coordinate vector} of~\wv\ with respect to~\cB}.
\end{theorem}
\begin{proof}  
Consider any vector \(\wv\in\WW\).
Since \(\{\hlist\vv k\}\) is a basis for the subspace~\WW, \wv~can be written as a linear combination of the basis vectors. 
Let two such linear combinations be
\begin{eqnarray*}&&
\wv=\lincomb c\vv k\,,
\\&&
\wv=\lincomb d\vv k\,.
\end{eqnarray*}
Subtract the second of these equations from the first, grouping common vectors:
\begin{equation*}
\ov=(c_1-d_1)\vv_1+(c_2-d_2)\vv_2+\cdots+(c_k-d_k)\vv_k\,.
\end{equation*}
Since \(\{\hlist\vv k\}\) is linearly independent, this equation implies all the coefficients in parentheses are zero:
\begin{equation*}
0=(c_1-d_1)=(c_2-d_2)=\cdots=(c_k-d_k).
\end{equation*}
That is, \(c_1=d_1\)\,, \(c_2=d_2\)\,, \ldots, \(c_k=d_k\)\,, and the two linear combinations are identical.
That is, there is exactly one way to write a vector \(\wv\in\WW\) as a {linear combination} of the {basis} vectors.
\end{proof}


\begin{example} \label{eg:} 
\begin{enumerate}
\item 
\newcommand{\ppoint}[3]{%
    \pgfmathparse{#1*2-#2*1}\let\h\pgfmathresult
    \pgfmathparse{-#1*1+#2*2}\let\v\pgfmathresult
    \addplot[blue,mark=*,thick,quiver={u=\h,v=\v},-stealth] coordinates {(0,0)};
    \edef\tempe{%
    \noexpand\node[right] at (axis cs:\h,\v) {$\noexpand\vec #3$};
    }\tempe
    }%
\newcommand{\cbase}[1]{\begin{tikzpicture}
\begin{axis}[small,font=\footnotesize
  , axis lines=none
  , axis equal image
  , view={0}{90}
  ,xmax=7.9,ymax=5.5,xmin=-7.1,ymin=-5.5
  ]
\ifcase#1
\or\addplot3[mesh,red,samples=9,domain=-4:4,dotted] (2*x-y,-x+2*y,0);
\fi
\addplot[red,quiver={u=2,v=-1},-stealth,thick] coordinates {(0,0)};
\node[right] at (axis cs:2,-1) {$\vv_1$};
\addplot[red,quiver={u=-1,v=2},-stealth,thick] coordinates {(0,0)};
\node[above] at (axis cs:-1,2) {$\vv_2$};
\ppoint32{a}
\ppoint1{-2}{b}
\ppoint{-2}{0.5}{c}
\ppoint{-1}{-1}{d}
\end{axis}
\end{tikzpicture}}%
Consider the diagram of six labelled vectors drawn below.\par
\begin{tabular}{@{}c@{\ }c@{}}
\cbase0
&\parbox[b]{0.4\linewidth}{
Estimate the coordinates of the four shown vectors~\av, \bv, \cv\ and~\dv\ in the shown basis \(\cB=\{\vv_1,\vv_2\}\).}
\end{tabular}
\begin{solution}
Draw in a grid corresponding to multiples of~\(\vv_1\) and~\(\vv_2\) in both directions, and parallel to~\(\vv_1\) and~\(\vv_2\),  as shown below.
Then from the grid, estimate that \(\av\approx3\vv_1+2\vv_2\) hence the coordinates \([\av]_\cB\approx(3,2)\).
\par
\begin{tabular}{@{}c@{\ }c@{}}
\cbase1
&\parbox[b]{0.4\linewidth}{
Similarly, \(\bv\approx \vv_1-2\vv_2\) hence the coordinates \([\bv]_\cB\approx(1,-2)\).
Also, \(\cv\approx -2\vv_1+0.5\vv_2\) hence the coordinates \([\cv]_\cB\approx(-2,0.5)\).
And lastly,  \(\dv\approx -\vv_1-\vv_2\) hence the coordinates \([\dv]_\cB\approx(-1,-1)\).
}
\end{tabular}
\end{solution}

\item 
\renewcommand{\ppoint}[3]{%
    \pgfmathparse{#1*1-#2*2}\let\h\pgfmathresult
    \pgfmathparse{-#1*2-#2*2}\let\v\pgfmathresult
    \addplot[blue,mark=*,thick,quiver={u=\h,v=\v},-stealth] coordinates {(0,0)};
    \edef\tempe{%
    \noexpand\node[right] at (axis cs:\h,\v) {$\noexpand\vec #3$};
    }\tempe
    }%
\renewcommand{\cbase}[1]{\begin{tikzpicture}
\begin{axis}[small,font=\footnotesize
  , axis lines=none
  , axis equal image
  , view={0}{90}
  ,xmax=7.9,ymax=5.5,xmin=-7.1,ymin=-5.5
  ]
\ifcase#1
\or\addplot3[mesh,red,samples=9,domain=-4:4,dotted] (1*x-2*y,-2*x-2*y,0);
\fi
\addplot[red,quiver={u=1,v=-2},-stealth,thick] coordinates {(0,0)};
\node[below] at (axis cs:1,-2) {$\wv_1$};
\addplot[red,quiver={u=-2,v=-2},-stealth,thick] coordinates {(0,0)};
\node[below] at (axis cs:-2,-2) {$\wv_2$};
\ppoint1{-1.5}{a}
\ppoint3{-0.5}{b}
\ppoint{-2.5}{1}{c}
\ppoint{0}{0.5}{d}
\end{axis}
\end{tikzpicture}}%
Consider the same four vectors but with a pair of different basis vectors:  let's see that although the vectors are the same, the coordinates in the different basis are different.\par
\begin{tabular}{@{}c@{\ }c@{}}
\cbase0
&\parbox[b]{0.4\linewidth}{
Estimate the coordinates of the four shown vectors~\av, \bv, \cv\ and~\dv\ in the shown basis \(\cW=\{\wv_1,\wv_2\}\).}
\end{tabular}
\begin{solution}
Draw in a grid corresponding to multiples of~\(\wv_1\) and~\(\wv_2\) in both directions, and parallel to~\(\wv_1\) and~\(\wv_2\),  as shown below.
Then from the grid, estimate that \(\av\approx\wv_1-1.5\wv_2\) hence the coordinates \([\av]_\cW\approx(1,-1.5)\).
\par
\begin{tabular}{@{}c@{\ }c@{}}
\cbase1
&\parbox[b]{0.4\linewidth}{
Similarly, \(\bv\approx 3\wv_1-0.5\wv_2\) hence the coordinates \([\bv]_\cW\approx(3,-0.5)\).
Also, \(\cv\approx -2.5\wv_1+\wv_2\) hence the coordinates \([\cv]_\cW\approx(-2.5,1)\).
And lastly,  \(\dv\approx 0.5\wv_2\) hence the coordinates \([\dv]_\cW\approx(0,0.5)\).
}
\end{tabular}
\end{solution}

\end{enumerate}
\end{example}



\begin{example} \label{eg:}
Let the basis \(\cB=\{\vv_1,\vv_2,\vv_3\}\) for the three given vectors \(\vv_1=(-1,1,-1)\), \(\vv_2=(1,-2,0)\) and \(\vv_3=(0,4,5)\) (specified in the \idx{standard basis}~\cE\ of the \idx{standard unit vector}s~\(\ev_1\), \(\ev_2\) and~\(\ev_3\)).
\begin{enumerate}
\item What is the vector with coordinates \([\av]_\cB=(3,-2,1)\)?
\begin{solution} 
\(\av=3\vv_1-2\vv_2+\vv_3\) which has standard coordinates 
\([\av]_\cE=3(-1,1,-1)-2(1,-2,0)+(0,4,5)=(-5,11,2)\).
\end{solution}

\item What is the vector with coordinates \([\bv]_\cB=(-1,1,1)\)?
\begin{solution} 
\(\bv=-\vv_1+\vv_2+\vv_3\) which has standard coordinates 
\([\bv]_\cE=-(-1,1,-1)+(1,-2,0)+(0,4,5)=(2,1,6)\).
\end{solution}

\item What are the coordinates in the basis~\cB\ of the vector \(\cv=(-1,3,3)\) in the \idx{standard basis}~\cE?
\begin{solution} 
We seek coordinate values \(c_1,c_2,c_3\) such that \(\cv=c_1\vv_1+c_2\vv_2+c_3\vv_3\)\,. 
Expressed in the standard basis this equation is
\begin{equation*}
\begin{bmatrix} -1\\3\\3 \end{bmatrix}=
\begin{bmatrix} -1\\1\\-1 \end{bmatrix}c_1+
\begin{bmatrix} 1\\-2\\0 \end{bmatrix}c_2+
\begin{bmatrix} 0\\4\\5 \end{bmatrix}c_3\,.
\end{equation*}
A small system like this we solve by hand (recall section~\ref{sec:amss}): write in component form as
\begin{eqnarray*}
&&\left\{\begin{array}{r@{\ }r@{\ }r@{{}={}}l}  
-c_1&+c_2&&-1 \\
c_1&-2c_2&+4c_3&3 \\
-c_1&&+5c_3&3
\end{array}\right.
\\&&(\text{add 1st row to 2nd and take from 3rd})
\\&&\left\{\begin{array}{r@{\ }r@{\ }r@{{}={}}l}  
-c_1&+c_2&&-1 \\
&-c_2&+4c_3&2 \\
&-c_2&+5c_3&4
\end{array}\right.
\\&&(\text{subtract 2nd row from 3rd})
\\&&\left\{\begin{array}{r@{\ }r@{\ }r@{{}={}}l}  
-c_1&+c_2&&-1 \\
&-c_2&+4c_3&2 \\
&&c_3&2
\end{array}\right.
\end{eqnarray*}
Solving this triangular system gives \(c_3=2\)\,, \(c_2=4c_3-2=6\)\,, and \(c_1=c_2+1=7\)\,.
Thus the coordinates \([\cv]_\cB=(7,6,2)\) in the basis~\cB.
\end{solution}


\item What are the coordinates in the basis~\cB\ of the vector \(\dv=(-3,2,0)\) in the \idx{standard basis}~\cE?
\begin{solution} 
We seek coordinate values \(d_1,d_2,d_3\) such that \(\dv=d_1\vv_1+d_2\vv_2+d_3\vv_3\)\,. 
Expressed in the standard basis this equation is
\begin{equation*}
\begin{bmatrix} -3\\2\\0 \end{bmatrix}=
\begin{bmatrix} -1\\1\\-1 \end{bmatrix}d_1+
\begin{bmatrix} 1\\-2\\0 \end{bmatrix}d_2+
\begin{bmatrix} 0\\4\\5 \end{bmatrix}d_3\,.
\end{equation*}
To solve this system with \script\ (procedure~\ref{pro:unisol}), enter the matrix (easiest by transposing rows of~\(\vv_1\), \(\vv_2\) and~\(\vv_3\)) and the standard coordinates of~\dv:
\begin{verbatim}
A=[-1 1 -1
 1 -2 0
 0 4 5]'
d=[-3;2;0]
\end{verbatim}
\setbox\ajrqrbox\hbox{\qrcode{% new basis
A=[-1 1 -1
 1 -2 0
 0 4 5]'
d=[-3;2;0]
rcond(A)
dB=A\slosh d
}}%
\marginpar{\usebox{\ajrqrbox}}%
Then compute the coordinates~\([\dv]_\cB\) with \verb|dB=A\d| to determine \([\dv]_\cB=(20,17,4)\).

But remember, before using~\verb|A\| always first check \verb|rcond(A)| which here is the poor~\(0.0053\) (Procedure~\ref{pro:unisol}).
Interestingly, this poor small value of \verb|rcond| indicates that although the basis vectors in~\cB\ are linearly independent, they are `only just' linearly independent.
A small change or error might make them linearly dependent and thus \cB~be ruined as a basis for~\(\RR^3\).
The poor \verb|rcond| indicates that~\cB\ is a poor basis in practice.
\end{solution}
\end{enumerate}
\end{example}





\begin{example} \label{eg:}
You are given a \idx{basis} \(\cW=\{\wv_1,\wv_2,\wv_3\}\) for a 3D subspace~\WW\ of~\(\RR^5\) where the three basis vectors are
\(\wv_1=(1,3,-4,-3,3)\),
\(\wv_2=(-4,1,-2,-4,1)\), and
\(\wv_3=(-1,1,0,2,-3)\) (in the \idx{standard basis}~\cE).
\begin{enumerate}
\item What are the coordinates in the standard basis of the vector \(\av=2\wv_1+3\wv_2+\wv_3\)?
\begin{solution} 
In the standard basis
\begin{equation*}
[\av]_\cE=2\begin{bmatrix} 1\\3\\-4\\-3\\3 \end{bmatrix}
+3\begin{bmatrix} -4\\1\\-2\\-4\\1 \end{bmatrix}
+\begin{bmatrix} -1\\1\\0\\2\\-3 \end{bmatrix}
=\begin{bmatrix} -11\\ 10\\-14\\-16\\  6 \end{bmatrix}.
\end{equation*}
\end{solution}

\item What are the coordinates in the basis~\cW\ of the vector \(\bv=(-1,2,-6,-11,10)\) (in the standard coordinates~\cE).
\begin{solution} 
We need to find coefficients \(c_1,c_2,c_3\) such that \(\bv=c_1\wv_1+c_2\wv_2+c_3\wv_3\)\,.
This forms the set of linear equations
\begin{equation*}
\begin{bmatrix}-1\\2\\-6\\-11\\10 \end{bmatrix}
=\begin{bmatrix} 1\\3\\-4\\-3\\3 \end{bmatrix}c_1
+\begin{bmatrix} -4\\1\\-2\\-4\\1 \end{bmatrix}c_2
+\begin{bmatrix} -1\\1\\0\\2\\-3 \end{bmatrix}c_3\,.
\end{equation*}
Form as the matrix-vector system
\begin{equation*}
\begin{bmatrix}1&-4&-1
\\ 3&1&1
\\-4&-2&0
\\-3&-4&2
\\ 3&1&-3\end{bmatrix}
\begin{bmatrix} c_1\\c_2\\c_3 \end{bmatrix}
=\begin{bmatrix}-1\\2\\-6\\-11\\10 \end{bmatrix},
\end{equation*}
and perhaps solve with \script.
Since there are more equations than unknowns, we should use an \svd\ in order to check the system is consistent, namely, to check that \(\bv\in\WW\).
\begin{enumerate}
\item Code the matrix and the vector:
\begin{verbatim}
W=[1 -4 -1
 3 1 1
 -4 -2 0
 -3 -4 2
 3 1 -3]
b=[-1;2;-6;-11;10]
\end{verbatim}
\setbox\ajrqrbox\hbox{\qrcode{% new basis
W=[1 -4 -1
 3 1 1
 -4 -2 0
 -3 -4 2
 3 1 -3]
b=[-1;2;-6;-11;10]
[U,S,V]=svd(W)
z=U'*b
y=z(1:3)./diag(S)
bw=V*y
}}%
\marginpar{\usebox{\ajrqrbox}}%
\item Then obtain an \svd\ with \verb|[U,S,V]=svd(W)| \twodp
\begin{verbatim}
U =
   0.18  -0.88   0.04  -0.24  -0.37
  -0.32  -0.09   0.65   0.62  -0.28
   0.51   0.11  -0.46   0.56  -0.44
   0.64  -0.18   0.34   0.22   0.63
  -0.44  -0.42  -0.49   0.44   0.45
S =
   8.18      0      0
      0   4.52      0
      0      0   3.11
      0      0      0
      0      0      0
V =
  -0.74  -0.51   0.43
  -0.62   0.77  -0.15
   0.26   0.38   0.89
\end{verbatim}
The three non-zero singular values establish that the three vectors in the basis~\cW\ are indeed linearly independent (and since no singular value is small, then the vectors are robustly linearly independent).

\item Find \(\zv=\tr U\bv\) with \verb|z=U'*b| to get
\begin{verbatim}
z =
  -15.3413
   -2.2284
   -4.6562
   -0.0000
    0.0000
\end{verbatim}
The last two values of~\zv\ being zero confirm the system of equations is consistent and so vector~\(\bv\) is in the range of~\cW, that is, \bv~is in the subspace~\WW.

\item Find \(y_j=z_j/\sigma_j\) with \verb|y=z(1:3)./diag(S)| to get
\begin{verbatim}
y =
  -1.8761
  -0.4929
  -1.4958
\end{verbatim}

\item Lastly, find the coefficients \([\bv]_\cW=V\yv\) with \verb|bw=V*y| to get
\begin{verbatim}
bw =
   1.00000
   1.00000
  -2.00000
\end{verbatim}
\end{enumerate}
That is, \([\bv]_\cW=(1,1,-2)\).

That \([\bv]_\cW\) has three components and \([\bv]_\cE\) has five components is not a contradiction.  
The difference in components occurs because the subspace~\WW\ is 3D but lies in~\(\RR^5\).  
Using the basis~\cW\ implicitly builds in the information that the vector~\bv\ is in a lower dimensional space, and so needs fewer components.
\end{solution}

\end{enumerate}
\end{example}





\begin{comment} 
Optionally: Gram--Schmidt (probably not here as would confuse the aims of lin indep, next section); and something on change of basis, recalling quadratics.
\end{comment}








\subsubsection{Revisit unique solutions}
Lastly, with all these extra concepts of determinants, eigenvalues,  linear independence and a basis, we now revisit the issue of when there is a unique solution to a set of linear equations.


\begin{theorem}[Unique Solutions: version~3] \label{thm:ftim3} 
Let \(A\) be an \(n\times n\) \idx{square matrix}.  
Extending Theorem~\ref{thm:ftim2}, the following statements are equivalent:
\begin{enumerate}
\item\label{thm:ftim3i} \(A\) is \idx{invertible};
\item\label{thm:ftim3ii} \(A\xv=\bv\) has a \idx{unique solution} for every \(\bv\in\RR^n\);
\item\label{thm:ftim3iii} \index{homogeneous}\(A\xv=\ov\) has only the zero solution;
\item\label{thm:ftim3iv} all \(n\)~\idx{singular value}s of~\(A\) are nonzero;
\item\label{thm:ftim3v} \(\rank A=n\)\,;
\item\label{thm:ftim3vi} \(\nullity A=0\)\,;
\item\label{thm:ftim3vii} the \idx{column vector}s of~\(A\) span~\(\RR^n\);
\item\label{thm:ftim3viii} the \idx{row vector}s of~\(A\) span~\(\RR^n\).
\item\label{thm:ftim3ix} \(\det A\neq 0\)\,;
\item\label{thm:ftim3x} \(0\) is not an eigenvalue of~\(A\);
\item\label{thm:ftim3xi} the \(n\) column vectors of~\(A\) are linearly independent;
\item\label{thm:ftim3xii} the \(n\) row vectors of~\(A\) are linearly independent.
\end{enumerate}
\end{theorem}
\begin{proof} 
Recall Theorems~\ref{thm:ftim1} and~\ref{thm:ftim2} established the equivalence of \ref{thm:ftim3i}--\ref{thm:ftim3viii}, and Theorem~\ref{thm:detinv} proved the equivalence of \ref{thm:ftim3i} and~\ref{thm:ftim3ix}.
To establish that Property~\ref{thm:ftim3x} is equivalent to~\ref{thm:ftim3ix}, recall Theorem~\ref{thm:charpolyc} proved that \(\det A\) equals the product of the eigenvalues of~\(A\). 
Hence  \(\det A\) is not zero if and only if all the eigenvalues are non-zero.  

Lastly, Property~\ref{thm:ftim3vii} says the \(n\)~column vectors span~\(\RR^n\), so they must be a basis, and hence linearly independent.
Conversely, if the \(n\)~columns of~\(A\) are linearly independent then they must span~\(\RR^n\).  
Hence Property~\ref{thm:ftim3xi} is equivalent to~\ref{thm:ftim3vii}.
Similarly  for Property~\ref{thm:ftim3viii} and the row vectors of~\ref{thm:ftim3xii}.
\end{proof}









\subsection{Exercises}



\begin{exercise} \label{ex:} 
By inspection or basic arguments, decide wether the following sets of vectors are linearly dependent, or linearly independent.  Give reasons.
%\begin{verbatim}
%m=ceil(0.5+3*rand), n=ceil(1+2.5*rand), A=0+round(randn(m,n)*2), svd(A)
%\end{verbatim}

\begin{enumerate}
\item \(\{(-2,3,3), (-1,2,-1)\}\)
\answer{lin.\ indep.}

\item \(\{(0,2), (2,-2), (0,-1)\}\)
\answer{lin.\ dep.}

\item \(\{(-3,0,-3), (3,2,-2)\}\)
\answer{lin.\ indep.}

\item \(\{(0,2,2)\}\)
\answer{lin.\ indep.}

\item \(\{(-2,0,-1), (0,-2,2), (2,0,1)\}\)
\answer{lin.\ dep.}

\item \(\{(0,3,-2), (-2,-1,1), (1,-2,-4)\}\)
\answer{lin.\ indep.}

\item \(\{(-2,1),,(0,0)\}\)
\answer{lin.\ dep.}

\item \(\{(-2,-1,1), (-2,-2,2), (2,-1,-1), (2,-2,-2)\}\)
\answer{lin.\ dep.}

\item \(\{(2,4), (1,2)\}\)
\answer{lin.\ dep.}

\item \(\{( 1,-2,3), (1,1,2)\}\)
\answer{lin.\ indep.}

\end{enumerate}
\end{exercise}







\begin{exercise} \label{ex:} 
Compute an \svd\ to decide whether the following sets of vectors are linearly dependent, or linearly independent.  Give reasons.
%\begin{verbatim}
%m=ceil(1+3*rand), n=floor(m+2.5*rand), for j=1:(m*n)^2,A=0+round(randn(m,n)*3); sv=svd(A); if min(sv)<1e-7,break,end,end, A=A, sv=sv
%\end{verbatim}
\begin{parts}
\item \(\begin{bmatrix} 5\\0\\-1 \end{bmatrix}, \begin{bmatrix}-2\\5\\-1 \end{bmatrix}\)
\answer{lin.\ indep.}

\item \(\begin{bmatrix} -2\\-1\\1 \end{bmatrix}, \begin{bmatrix} 3\\-4\\-1 \end{bmatrix}, \begin{bmatrix} 5\\-3\\-2 \end{bmatrix}\)
\answer{lin.\ dep.}

\item \(\begin{bmatrix} 2\\-4\\1\\6 \end{bmatrix}, \begin{bmatrix} -1\\1\\10\\-5 \end{bmatrix}, \begin{bmatrix}1\\6\\-2\\4 \end{bmatrix}\)
\answer{lin.\ indep.}

\item \(\begin{bmatrix} 1\\0\\-5\\2 \end{bmatrix}, \begin{bmatrix}-1\\0\\-2\\4 \end{bmatrix}, \begin{bmatrix}-1\\3\\1\\1 \end{bmatrix}, \begin{bmatrix} 3\\4\\-4\\-4 \end{bmatrix}\)
\answer{lin.\ dep.}

\item \(\begin{bmatrix} 0\\-2\\2\\0 \end{bmatrix}, \begin{bmatrix}-1\\-4\\-6\\-1 \end{bmatrix}, \begin{bmatrix} 4\\3\\-5\\0 \end{bmatrix}\)
\answer{lin.\ indep.}

\item \(\begin{bmatrix} 0\\3\\2\\0 \end{bmatrix}, \begin{bmatrix}-3\\-3\\1\\-2 \end{bmatrix}, \begin{bmatrix}-3\\0\\3\\-2 \end{bmatrix}\)
\answer{lin.\ dep.}

\item \(\begin{bmatrix} -3\\2\\6\\3\\-2 \end{bmatrix}, \begin{bmatrix}-3\\-3\\1\\6\\-4 \end{bmatrix}, \begin{bmatrix} 3\\-2\\-2\\4\\-1 \end{bmatrix}, \begin{bmatrix} 1\\-2\\-2\\3\\2 \end{bmatrix}\)
\answer{lin.\ indep.}

\item \(\begin{bmatrix} 3\\0\\4\\1\\2 \end{bmatrix}, \begin{bmatrix}-2\\1\\2\\2\\2 \end{bmatrix}, \begin{bmatrix} 2\\1\\0\\0\\0 \end{bmatrix}, \begin{bmatrix}-9\\-1\\-2\\1\\0 \end{bmatrix}\)
\answer{lin.\ dep.}

\end{parts}
\end{exercise}




\begin{exercise} \label{ex:lindeplc} 
Prove the particular case of Theorem~\ref{thm:lindeplc}, namely that a set of two vectors \(\{\vv_1,\vv_2\}\) is linearly dependent if and only if one of the vectors is a scalar multiple of the other.
\end{exercise}




\begin{exercise} \label{ex:} 
Prove that every (non-empty) subset of a linearly independent set is also linearly independent.  (Perhaps use contradiction.)
\end{exercise}





\begin{exercise} \label{ex:} 
Let \(\{\hlist\vv m\}\) be a linearly independent set of vectors in~\(\RR^n\).
Given that a vector \(\uv=\lincomb c\vv m\) with coefficient \(c_1\neq0\)\,, prove that the set \(\{\hlist[2]\vv m,\uv\}\) is linearly independent.
\end{exercise}




\begin{exercise} \label{ex:} 
For each of the following systems of equations find by hand two different bases for their solution set (among the infinitely many bases that are possible).  
Show your working.
% format rat
% A=0+round(randn(2,3)*3), v=null(A); v=v*diag(1./max(abs(v)))
\begin{enumerate}
\item \(-x-5y=0\) and \(y-3z=0\)
\answer{Two possibilities are 
\(\{(-1,1/5,1/15)\}\) and 
\(\{(15,-3,-1)\}\).}

\item \(6x+4y+2z=0\) and \(-2x-y-2z=0\)
\answer{Two possibilities are 
\(\{(-3/4,1,1/4)\}\) and 
\(\{(3,-4,-1)\}\).}

\item \(-2y-z+2=0\) and \(3x+4z=0\)
\answer{Does not have a solution subspace.}

\item \(-7x+y-z=0\) and \(-3x+2-2z=0\)
\answer{Two possibilities are 
\(\{(0,1,1)\}\) and 
\(\{(0,-2,-2)\}\).}

% A=quads(ceil(6*rand),1:3).*sign(randn(1,3)), v=null(A); v=v*diag(1./max(abs(v))), v*round(randn(2)*2)

\item \(x+2y+2z=0\)
\answer{Two possibilities are 
\(\{(-2,2,-1),(-2,-1,2)\}\) and 
\(\{(0,1,-1),(-8,11,-7)\}\).}

\item \(2x+0y-4z=0\)
\answer{Two possibilities are 
\(\{(0,1,0),(2,0,1)\}\) and 
\(\{(2,-4,1),(4,1,2)\}\).}

\item \(-2x+3y-6z=0\)
\answer{Two possibilities are 
\(\{(1/2,1,1/3),(-1,1/3,1/2)\}\) and 
\(\{(3,-8,-5),(12,10,1)\}\).}

\item \(9x+4y-9z=6\)
\answer{Does not have a solution subspace.}

\end{enumerate}
\end{exercise}







\begin{exercise} \label{ex:} 
Use Procedure~\ref{pro:bfe} to compute in \script\ an \idx{orthonormal basis} for all solutions to each of the following systems of equations.
%\begin{verbatim}
%format short,for i=1:9999,A=0+round(randn(3,3+ceil(2*rand))*4); if min(svd(A))<1e-7, A=A, break, end, end
%format bank, [u,s,v]=svd(A); sv=diag(s), vt=v'
%\end{verbatim}

\begin{enumerate}
\item \(2x-6y-9z=0\), \(2x-2z=0\), \(-x+z=0\)
\answer{\(\{(0.55,-0.64,0.55)\}\) \twodp}

\item \(3x-3y+8z=0\), \(-2x-4y+2z=0\), \(-4x+y-7z=0\)
\answer{\(\{(0.67,-0.57,-0.47)\}\) \twodp}

\item \(-2x+2y-2z=0\), \(x+3y-z=0\), \(3x+3y=0\)
\answer{\(\{(-0.41,0.41,0.82)\}\) \twodp}

\item \(2w+x+2y+z=0\), \(w+4x-4y-z=0\), \(3w-2x+5y=0\), \(2w-x+y-2z=0\)
\answer{\(\{(-0.50,0.50,0.50,-0.50)\}\) \twodp}

\item \(5w+y-3z=0\), \(-5x-5y=0\), \(-3x-y+4z=0\), \(3x+y-4z=0\)
\answer{\(\{(-0.32,-0.63,0.63,-0.32)\}\) \twodp}

\item \(-w-2y+4z=0\), \(2w+2y+2z=0\), \(-2w+3x+y+z=0\), \(-w+x-y+5z=0\)
\answer{\(\{(0.61,0.61,-0.51,-0.10)\}\) \twodp}

\item \(-2w+x+2y-6z=0\), \(-2w+3x+4y=0\), \(-2w+2x+3x-3z=0\)
\answer{\(\{(0.76,-0.27,0.58,-0.10), (-0.51,-0.78,0.33,0.15)\}\) \twodp}

\item \(-w-2x-3y+2z=0\), \(2x+2y-2z=0\), \(-w-3x-4y+3z=0\)
\answer{\(\{(0.62,0.45,-0.62,-0.17), (-0.13,0.63,0.13,0.76)\}\) \twodp}

\end{enumerate}
\end{exercise}






\begin{exercise} \label{ex:lenlam} 
Recall that Theorem~\ref{thm:lenlam} establishes there are at most \(n\)~eigenvalues of an \(n\times n\) symmetric matrix.
Adapt the proof of that theorem, using linear independence, to prove there are at most \(n\)~eigenvalues of an \(n\times n\) non-symmetric matrix.
(This is an alternative to the given proof of Theorem~\ref{thm:geecp}.)
%\begin{proof} 
%Invoke the \idx{pigeonhole principle} and \idx{contradiction}. 
%Assume there are more than \(n\)~distinct eigenvalues, then there would be more than \(n\)~eigenvectors corresponding to distinct eigenvalues.
%Theorem~\ref{thm:indepev} asserts all such eigenvectors are linearly independent. 
%But there cannot be more than \(n\)~vectors in a linearly independent set in~\(\RR^n\) (Theorem~\ref{thm:mgtnli}).
%Hence the assumption is wrong and there cannot be any more than \(n\)~distinct eigenvalues.
%\end{proof}
\end{exercise}




\begin{exercise} \label{ex:} 
For each diagram, estimate roughly the components of each of the four vectors~\av, \bv, \cv\ and~\dv\ in the basis \(\cP=\{\pv_1,\pv_2\}\).
\newcommand{\ppoint}[3]{%
    \pgfmathparse{#1*\px+#2*\qx}\let\h\pgfmathresult
    \pgfmathparse{#1*\py+#2*\qy}\let\v\pgfmathresult
    \addplot[blue,mark=*,thick,quiver={u=\h,v=\v},-stealth] coordinates {(0,0)};
    \edef\tempe{%
    \noexpand\node[right] at (axis cs:\h,\v) {$\noexpand\vec #3$};
    }\tempe
    }%
\newcommand{\cbase}[9]{\begin{tikzpicture}
\begin{axis}[small,font=\footnotesize
  , axis lines=none, unit vector ratio=1 1 1
  , view={0}{90}
  ]
\ifcase#1
\or\addplot3[mesh,red,samples=9,domain=-4:4,dotted] (\px*x+\qx*y,\py*x+\qy*y,0);
\fi
\addplot[red,quiver={u=\px,v=\py},-stealth,thick] coordinates {(0,0)};
\node[right] at (axis cs:\px,\py) {$\pv_1$};
\addplot[red,quiver={u=\qx,v=\qy},-stealth,thick] coordinates {(0,0)};
\node[right] at (axis cs:\qx,\qy) {$\pv_2$};
\ppoint{#2}{#3}{a}
\ppoint{#4}{#5}{b}
\ppoint{#6}{#7}{c}
\ppoint{#8}{#9}{d}
\end{axis}
\end{tikzpicture}
\answer{$(#2,#3)$, $(#4,#5)$, $(#6,#7)$, $(#8,#9)$}}%
\begin{enumerate}
\item \def\px{2} \def\py{-1} \def\qx{1} \def\qy{2} 
\cbase0{-2}{-3}{ 1}{-3}{-3}{-1}{ 3}{ 2}

\item \def\px{2} \def\py{-1} \def\qx{-1} \def\qy{3} 
\cbase0{-1.5}{ 0}{ 0.5}{-0.5}{-1}{ 2.5}{ 2}{ 0.5}

\item \def\px{-2} \def\py{1} \def\qx{1} \def\qy{-2} 
\cbase0{2.5}{ 0.5}{-1.5}{ 0}{ 1}{ 0.5}{ 1}{ 1.5}

\item \def\px{-3} \def\py{1} \def\qx{-1} \def\qy{3} 
\cbase0{-1}{-0.5}{ 1}{ 1}{-1}{0.5}{ 0.5}{0}

\item \def\px{0} \def\py{1} \def\qx{1} \def\qy{0} 
\cbase0{-0.9}{ 0.3}{ 0.6}{ 0.2}{-0.2}{-1.1}{ 0.7}{-0.6}

\item \def\px{-1} \def\py{-1} \def\qx{1} \def\qy{0} 
\cbase0{-0.9}{ 0.3}{ 0.6}{ 0.2}{-0.2}{-1.1}{ 0.7}{-0.6}

\item \def\px{1} \def\py{-3} \def\qx{1} \def\qy{1} 
\cbase0{0.7}{ 1.6}{ 1.2}{ 0.5}{-0.4}{-0.3}{ 1.4}{2}

\end{enumerate}
\end{exercise}




\begin{exercise} \label{ex:3basisv} 
Let the three given vectors \(\bv_1=(-1,1,-1)\), \(\bv_2=(1,-2,0)\) and \(\bv_3=(0,4,5)\) form a \idx{basis} \(\cB=\{\bv_1,\bv_2,\bv_3\}\) (where these vectors are specified in the \idx{standard basis}~\cE\ of the \idx{standard unit vector}s~\(\ev_1\), \(\ev_2\) and~\(\ev_3\)).
For each of the following vectors with specified coordinates in basis~\cB, what is the vector when written in the standard basis?
%\begin{verbatim}
%p=0+round(randn(1,3)*2)/2, pe=p*[-1 1 -1;1 -2 0;0 4 5]
%\end{verbatim}

\def\temp#1{% 
  \pgfmathparse{\value{enumii}}%
  \pgfmathprintnumberto[assume math mode=true]{\pgfmathresult}{\tempx}%
  \answer{\noexpand\setcounter{enumii}{\tempx}%
  $[\noexpand\evii]_{\noexpand\cE}=#1$}%
  }

\begin{parts}
\item \([\evii]_\cB=(1,-1,2)\)
\temp{(-2,11,9)}

\item \([\evii]_\cB=(0,2,3)\)
\temp{(2,8,15)}

\item \([\evii]_\cB=(1,-3,-2)\)
\temp{(-4,-1,-11)}

\item \([\evii]_\cB=(1,2,1)\)
\temp{(1,1,4)}

\item \([\evii]_\cB=(1/2,-1/2,1)\)
\temp{(-1,11/2,9/2)}

\item \([\evii]_\cB=(-1/2,1/2,-1/2)\)
\temp{(1,-7/2,-2)}

\item \([\evii]_\cB=(0,1/2,-1/2)\)
\temp{(1/2,-3,-5/2)}

\item \([\evii]_\cB=(-0.7,0.5,1.1)\)
\temp{(1.2,2.7,6.2)}

\item \([\evii]_\cB=(0.2,-0.1,0.9)\)
\temp{(-0.3,4.0,4.3)}

\item \([\evii]_\cB=(2.1,-0.2,0.1)\)
\temp{(-2.3,2.9,-1.6)}

\end{parts}
\end{exercise}





\begin{exercise} \label{ex:} 
Repeat Exercise~\ref{ex:3basisv} but with the three basis vectors \(\bv_1=(6,2,1)\), \(\bv_2=(-2,-1,-2)\) and \(\bv_3=(-3,-1,5)\).
% V=0+round(randn(3)*3),sv=svd(V)
\end{exercise}





\begin{exercise} \label{ex:2basev} 
Let the two given vectors \(\bv_1=(1,-2,2)\) and \(\bv_2=(1,-1,-1)\) form a basis \(\cB=\{\bv_1,\bv_2\}\) for the subspace~\BB\ of~\(\RR^3\) (specified in the \idx{standard basis}~\cE\ of the \idx{standard unit vector}s~\(\ev_1\), \(\ev_2\) and~\(\ev_3\)).
For each of the following vectors, specified in the standard basis~\cE, what is the vector when written in the basis~\cB? if possible.
%\begin{verbatim}
%B=[1 -2 2;1 -1 -1]'
%pb=0+round(randn(1,2)*3)/10, pe=pb*B'+(rand<0.33)*round(randn(1,3)), pbt=B\pe'
%\end{verbatim}

\def\temp#1{% 
  \pgfmathparse{\value{enumii}}%
  \pgfmathprintnumberto[assume math mode=true]{\pgfmathresult}{\tempx}%
  \answer{\noexpand\setcounter{enumii}{\tempx}%
  $[\noexpand\evii]_{\noexpand\cB}=#1$}%
  }

\begin{parts}
\item \([\evii]_{\cE}=(0,1,3)\)
\temp{(-1,1)}

\item \([\evii]_{\cE}=(-4,9,-11)\)
\temp{(-5,1)}

\item \([\evii]_{\cE}=(-4,7,-5)\)
\temp{(-3,-1)}

\item \([\evii]_{\cE}=(0,2,-4)\)
\answer{not in \cB}

\item \([\evii]_{\cE}=(0,-2,5)\)
\answer{not in \cB}

\item \([\evii]_{\cE}=(-2/3,0,8/3)\)
\temp{(2/3,-4/3)}

\item \([\evii]_{\cE}=( -8/3,19/3,-19/3)\)
\answer{not in \cB}

\item \([\evii]_{\cE}=(-10/3,5,-5/3)\)
\temp{(-5/3,-5/3)}

\item \([\evii]_{\cE}=(-0.3,0.4,0)\)
\temp{(-0.1,-0.2)}

\item \([\evii]_{\cE}=(0.5,-0.7,0.1)\)
\temp{(0.2,0.3)}

\end{parts}
\end{exercise}





\begin{exercise} \label{ex:} 
Repeat Exercise~\ref{ex:2basev} but with the two basis vectors \(\bv_1=(-2,3,-1)\) and \(\bv_2=(0,-1,3)\).
\end{exercise}






\begin{exercise} \label{ex:3base5v} 
Let the three vectors \(\bv_1=(-1,-2,5,3,-2)\), \(\bv_2=(-2,-2,2,-1,-1)\) and \(\bv_3=(-4,6,-4,2,-1)\) form a \idx{basis} \(\cB=\{\bv_1,\bv_2,\bv_3\}\) for the subspace~\BB\ in~\(\RR^5\) (specified in the \idx{standard basis}~\cE).
\setbox\ajrqrbox\hbox{\qrcode{% basis
B=[-1 -2 -4
 -2 -2 6
 5 2 -4
 3 -1 2
 -2 -1 -1]
}}\marginpar{\usebox{\ajrqrbox}}%
For each of the following vectors, use \script\ to find the requested coordinates, if possible.
%\begin{verbatim}
%B=0+round(randn(5,3)*3), svb=svd(B)
%pb=0+round(randn(1,3)*3)/1, pe=pb*B'+(rand<0.33)*round(randn(1,5)), pbt=B\pe'
%\end{verbatim}
\def\temp#1#2{% 
  \pgfmathparse{\value{enumii}}%
  \pgfmathprintnumberto[assume math mode=true]{\pgfmathresult}{\tempx}%
  \answer{\noexpand\setcounter{enumii}{\tempx}%
  $[\noexpand\evii]_{\noexpand#1}=#2$}%
  }
\begin{enumerate}
\item Find \([\evii]_\cE\) when \([\evii]_\cB=(2,2,-1)\).
\temp{\cE}{(-2,-14,18,2,-5)}

\item Find \([\evii]_\cE\) when \([\evii]_\cB=(-5,0,-2)\).
\temp{\cE}{(13,-2,-17,-19,12)}

\item Find \([\evii]_\cE\) when \([\evii]_\cB=(0,3,3)\).
\temp{\cE}{(-18,12,-6,3,-6)}

\item Find \([\evii]_\cE\) when \([\evii]_\cB=(-1,5,0)\).
\temp{\cE}{(-9,-8,5,-8,-3)}

\item Find \([\evii]_\cB\) when \([\evii]_\cE=(-31,26,-5,19,-14)\).
\temp{\cB}{(3,2,6)}

\item Find \([\evii]_\cB\) when \([\evii]_\cE=(-1,6,4,14,-5)\).
\answer{not in \BB}

\item Find \([\evii]_\cB\) when \([\evii]_\cE=(-21,18,-7,9,-8)\).
\temp{\cB}{(1,2,4)}

\item Find \([\evii]_\cB\) when \([\evii]_\cE=(-0.2,-0.6,1.8,1.3,-0.7)\).
\temp{\cB}{(0.4,-0.1,0)}

\item Find \([\evii]_\cB\) when \([\evii]_\cE=(0.7,-0.4,-0.3,-0.3,1.2)\).
\answer{not in \BB}

\item Find \([\evii]_\cB\) when \([\evii]_\cE=(4.8,-3.8,0.8,-2.6,2.1)\).
\temp{\cB}{(-0.4,-0.4,-0.9)}

\end{enumerate}
\end{exercise}




\begin{exercise} \label{ex:} 
Repeat Exercise~\ref{ex:3base5v} but with basis vectors \(\bv_1=(-3,8,-9,-1,1)\), \(\bv_2=(10,-20,14,-7,2)\) and \(\bv_3=(-1,-2,5,3,-2)\) (specified in the \idx{standard basis}~\cE).
%for i=1:9999,P=round(randn(3)*2); if abs(det(P))==1, P=P, break, end, end
\setbox\ajrqrbox\hbox{\qrcode{% basis
B=[-3 10 -1
 8 -20 -2
 -9 14 5
 -1 -7 3
 1 2 -2]
}}\marginpar{\usebox{\ajrqrbox}}
\end{exercise}




\begin{comment}%{ED498555.pdf}
why, what caused X?
how did X occur?
what-if? what-if-not?
how does X compare with Y?
what is the evidence for X?
why is X important?
\end{comment}




\index{linearly dependent|)}
\index{linearly independent|)}
\index{linear dependence|)}
\index{linear independence|)}


