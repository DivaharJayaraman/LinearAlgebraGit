%!TEX root = ../larxxia.tex

\chapter{Eigenvalues and eigenvectors in general}
\label{ch:gee}

\minitoc





\paragraph{Population modelling} \index{population modelling}
Suppose two species of animals interact: how do their populations evolve in time?  
Let \(y(t)\) and~\(z(t)\) be the number of female animals in each of the species at time~\(t\) in years (biologists usually just count females in population models as females usually determine reproduction). 
Modelling might deduce the populations interact according to the rule that the population one year later is \(y(t+1)=2y(t)-4z(t)\) and \(z(t+1)=-y(t)+2z(t)\): that is, if it was not for the other species, then for each species the number of females would both double every year (since then \(y(t+1)=2y(t)\) and \(z(t+1)=2z(t)\)); but the other species decreases each of these growths via the~\(-4z(t)\) and~\(-y(t)\) terms.  

Question: can we find special solutions in the form \((y,z)=\xv \lambda^{t}\) for some constant~\(\lambda\)?  
Substitute \(y=x_1\lambda^{t}\) and \(z=x_2\lambda^{t}\) into the  equations:
\begin{eqnarray*}
&&y(t+1)=2y(t)-4z(t)\,,\quad z(t+1)=-y(t)+2z(t)
\\&\iff& 
x_1 \lambda^{t+1}=2x_1\lambda^{t}-4x_2\lambda^{t}
\,,\quad
x_2 \lambda^{t+1}=-x_1\lambda^{t}+2x_2\lambda^{t}
\\&\iff& 2x_1-4x_2=\lambda x_1
\,,\quad
-x_1+2x_2=\lambda x_2
\end{eqnarray*}
after dividing by the factor~\(\lambda^{t}\) (assuming constant~\(\lambda\) is non-zero).
Forming as a matrix-vector equation these last two equations are
\begin{equation*}
\begin{bmatrix} 2&-4\\-1&2 \end{bmatrix}\xv=\lambda\xv\,.
\end{equation*}
That is, this substitution \((y,z)=\xv \lambda^{t}\) shows the question about solutions of the population equations reduces to solving \(A\xv=\lambda \xv\)\,, called an \idx{eigen-problem}.

The linear algebra of the eigen-problem will empower us to predict that the general solution for the population is, in terms of two constants~\(c_1\) and~\(c_2\), that one species has female population \(y(t)=2c_14^t+2c_2\) whereas the the second species has female population \(z(t)=-c_14^t+c_2\)\,.

%Modelling might deduce the populations interact according to the pair  of \idx{differential equation}s \(\dot y=y-4z\) and \(\dot z=-y+z\) (where the overdot denotes~\(d/dt\)); that is, if it was not for the other species the number of females would both grow according to \(\dot y=y\) and \(\dot z=z\)\,, but the other species decreases their growth via the~\(-2z\) and~\(-y\) terms.  
%
%Question: can we find special solutions in the form \((y,z)=\xv e^{\lambda t}\)?  
%Substitute \(y=x_1e^{\lambda t}\) and \(z=x_2e^{\lambda t}\) into the differential equations:
%\begin{eqnarray*}
%&&\dot y=y-4z\,,\quad \dot z=-y+z
%\\&\iff& 
%x_1\lambda e^{\lambda t}=x_1e^{\lambda t}-4x_2e^{\lambda t}
%\,,\quad
%x_2\lambda e^{\lambda t}=-x_1e^{\lambda t}+x_2e^{\lambda t}
%\\&\iff& x_1-4x_2=\lambda x_1
%\,,\quad
%-x_1+x_2=\lambda x_2
%\end{eqnarray*}
%after dividing by the exponential~\(e^{\lambda t}\) (it is always non-zero).
%Forming as a matrix-vector equation these last two equations are
%\begin{equation*}
%\begin{bmatrix} 1&-4\\-1&1 \end{bmatrix}\xv=\lambda\xv\,.
%\end{equation*}
%That is, this substitution \((y,z)=\xv e^{\lambda t}\) shows the question about solutions of differential equations reduces to solving \(A\xv=\lambda \xv\)\,, called an eigen-problem.





\paragraph{The basic eigen-problem}
Recall from Section~\ref{sec:iee} that the \idx{eigen-problem} equation \(A\xv=\lambda\xv\) is just asking can we find directions~\xv\ such that matrix~\(A\) acting on~\xv, resulting in~\(A\xv\), is in the same direction as~\xv, namely~\(\lambda\xv\) for some proportionality constant~\(\lambda\).
Now \(\xv=\ov\) is always a solution of the  equation \(A\xv=\lambda \xv\)\,.
Consequently, we are only interested in those values of the \idx{eigenvalue}~\(\lambda\) when non-zero solutions for the \idx{eigenvector}~\xv\ exist (as it is the directions which are of interest).
Rearranging the equation \(A\xv=\lambda\xv\) as the homogeneous system \((A-\lambda I)\xv=\ov\)\,, let's invoke properties of linear equations to solve the eigen-problem.
\begin{itemize}
\item Procedure~\ref{pro:eeh} establishes that one way to find the eigenvalues~\(\lambda\) is to solve the \idx{characteristic equation} \(\det(A-\lambda I)=0\)\,. 
\item Then for each eigenvalue, solving the homogeneous system \((A-\lambda I)\xv=\ov\) gives corresponding eigenvectors~\xv. 
\item The set of eigenvectors for a given eigenvalue forms a subspace called the \idx{eigenspace}~\(\EE_\lambda\) (Theorem~\ref{thm:espacedef}).

\end{itemize}






\paragraph{Three general difficulties in eigen-problems}
Recall that Section~\ref{sec:iee} introduced one way to visually estimate eigenvectors and eigenvalues of a given matrix~\(A\) \cite[]{Schonefeld1995}. 
The method is to plot many unit vectors~\xv, and at the end of each~\xv\ to adjoin the vector~\(A\xv\).
Since eigenvectors satisfy \(A\xv=\lambda\xv\) for some scalar eigenvalue~\(\lambda\), we visually identify eigenvectors as those~\xv\ which point in the same (or opposite) direction to~\(A\xv\).
Let's use this approach to identify three general difficulties.


\begin{enumerate}
\item 
In this first picture, for matrix
\( A=\scriptstyle\begin{bmatrix} 1&1\\\tfrac18&1 \end{bmatrix}\),
\marginpar{\eRose{1}1{0.125}{1}}%
the eigenvectors appear to be in directions \(\xv_1\approx\pm(0.9,0.3)\) and \(\xv_2\approx\pm(0.9,-0.3)\) corresponding to eigenvalues \(\lambda_1\approx1.4\) and \(\lambda_2\approx 0.6\)\,.
(Recall that scalar multiples of an eigenvector are always also eigenvectors, \S\ref{sec:iee}, so we always see \(\pm\)~pairs of eigenvectors in these pictures.)
These pairs of eigenvectors are not orthogonal, not at right angles---as happens for symmetric matrices (Theorem~\ref{thm:orthoevec}).
The lack of orthogonality in general means we soon generalise the concept of orthogonal sets of vectors to the new concept of \idx{linearly independent} sets (Section~\ref{sec:lisb}).

\item 
In this second case, for
\( A=\scriptstyle\begin{bmatrix} 0&1\\-1&\tfrac12 \end{bmatrix}\),
there appear to be no eigenvectors at all.
\marginpar{\eRose01{-1}{0.5}}%
No eigenvectors and eigenvalues is the answer if we require real answers.
However, in most applications we find it sensible to have complex valued eigenvalues and eigenvectors (Section~\ref{sec:eennm}).
So although we cannot see them graphically, for this matrix there are two \idx{complex eigenvalue}s and two families of \idx{complex eigenvector}s (analogous to those found in Example~\ref{eg:ccevals}). 
\footnote{In this second case the vectors~\(A\xv\) all appear to be pointing clockwise.  
Such a consistent rotation in~\(A\xv\) is characteristic of matrices with complex valued eigenvalues and eigenvectors.}

\item 
In this third case, for
\( A=\scriptstyle\begin{bmatrix} 1&1\\0&1 \end{bmatrix}\),
there appears to be only the vectors \(\xv=\pm(1,0)\), aligned along the horizontal axis, for which \(A\xv=\lambda\xv\).
\marginpar{\eRose1101}%
Whereas for symmetric matrices there were always two pairs, here we only appear to have one pair of eigenvectors (Theorem~\ref{thm:dimee}).
Such degeneracy occurs for matrices on the border between reality and complexity.

\end{enumerate}


The first problem of the general lack of orthogonality of the eigenvectors is most clearly seen in the case of triangular matrices (Definition~\ref{def:trim}).
The reason is  linked to Theorem~\ref{thm:rpdet:vi} that the determinant of a triangular matrix is simply the product of its diagonal entries.

\begin{example} \label{eg:}
Find algebraically the eigenvalues and eigenvectors of the triangular matrix \(A=\scriptstyle\begin{bmatrix} 2&1\\0&3 \end{bmatrix}\).
\begin{solution} 
Recall Procedure~\ref{pro:eeh}. 
\begin{enumerate}
\item Find all eigenvalues by solving the characteristic equation \(\det(A-\lambda I)=0\)\,. 
Here \(\det(A-\lambda I)=\det\scriptstyle\begin{bmatrix} 2-\lambda&1\\0&3-\lambda \end{bmatrix}\) which being a triangular matrix has determinant that is the product of the diagonals, namely \(\det(A-\lambda I)=(2-\lambda)(3-\lambda)\)\,.
This determinant is zero only for eigenvalues \(\lambda=2\) or~\(3\).
These are the diagonal entries in the triangular matrix.
\item For each eigenvalue, find corresponding eigenvectors by solving the system \((A-\lambda I)\xv=\ov\)\,.
\begin{itemize}
\item For \(\lambda=2\), the system is \({\scriptstyle\begin{bmatrix} 0&1\\0&1 \end{bmatrix}}\xv=\ov\) which requires \(x_2=0\)\,.  That is, all eigenvectors are~\(x_1(1,0)\).
\item For \(\lambda=3\), the system is \({\scriptstyle\begin{bmatrix} -1&1\\0&0 \end{bmatrix}}\xv=\ov\) which requires \(x_1=x_2\)\,.  That is, all eigenvectors are~\(x_2(1,1)\).
\end{itemize}
The eigenvectors corresponding to the different eigenvalues are not orthogonal as their dot product \((1,0)\cdot(1,1)=1+0=1\neq 0\)\,: instead they are at~\(45^\circ\) to each other.
\end{enumerate}
\end{solution}
\end{example}



\begin{theorem}[triangular matrices] \label{thm:trieig} 
The diagonal entries of a \idx{triangular matrix} are the only \idx{eigenvalue}s of the matrix.  
The corresponding \idx{eigenvector}s of distinct \idx{eigenvalue}s are \emph{generally} not \idx{orthogonal}.
\end{theorem}

\begin{proof} 
We detail only the case of upper triangular matrices as the argument is similar for lower triangular matrices. 
First establish that the diagonal entries are eigenvalues, and second prove there are no others.
Let \(\lambda\) be any value in the diagonal of the matrix~\(A\), and let \(k\)~be the smallest index such that \(a_{kk}=\lambda\) (this `smallest' caters for duplicated diagonal values).
Let's construct an eigenvector in the form \(\xv=(\hlist x{k-1},1,0,\ldots,0)\).  
Set \(x_k=1\) and \(x_j=0\) for \(j>k\)\,.
Then set 
\begin{eqnarray*}
x_{k-1}&=&-a_{k-1,k}x_k/(a_{k-1,k-1}-\lambda),
\\x_{k-2}&=&-(a_{k-2,k}x_k+a_{k-2,k-1}x_{k-1})/(a_{k-2,k-2}-\lambda),
\\&\vdots&
\\x_1&=&-(a_{1,k}x_k+a_{1,k-1}x_{k-1}+\cdots+a_{1,2}x_2)/(a_{1,1}-\lambda).
\end{eqnarray*}
Since \(k\)~is the smallest index for which \(\lambda=a_{k,k}\) none of the above expressions involve divisions by zero, and so all are well defined.
Rearranging the above equations shows that this vector~\xv\ satisfies, for  \(\lambda=a_{k,k}\)\,,
\begin{equation*}
\begin{array}{l@{}l@{}l@{}l@{}l@{}l}
(a_{1,1}-\lambda)x_1
&{}+a_{1,2}x_2
&{}+\cdots+a_{1,k-1}x_{k-1}
&{}+a_{1,k}x_k
&{}=0\,,
\\&\hspace{-1em}
(a_{2,2}-\lambda)x_2
&{}+\cdots+a_{2,k-1}x_{k-1}
&{}+a_{1,k}x_k
&{}=0\,,
\\&&\ddots&\vdots
\\&&\hspace{-1em}
(a_{k-1,k-1}-\lambda)x_{k-1}
&{}+a_{k-1,k}x_k
&{}=0\,,
\\&&&\hspace{-1em}
(a_{k,k}-\lambda)x_k
&{}=0\,;
\end{array}
\end{equation*}
that is, \((A-\lambda I)\xv=\ov\)\,.
Rearranging gives \(A\xv=\lambda\xv\) for non-zero eigenvector~\xv\ and corresponding eigenvalue~\(\lambda=a_{kk}\)\,.

Second, there can be no other eigenvalues.
Every eigenvalue has to have non-trivial solutions, eigenvectors~\xv, to \((A-\lambda I)\xv=\ov\)\,, which by Theorem~\ref{thm:detinv} requires \(\det(A-\lambda I)=0\)\,.
But matrix \((A-\lambda I)\) is upper triangular, as \(A\)~is upper triangular, so Theorem~\ref{thm:rpdet:vi} asserts the determinant
\begin{equation*}
\det(A-\lambda I)=(a_{1,1}-\lambda)(a_{2,2}-\lambda)
\cdots(a_{n,n}-\lambda)=0
\end{equation*}
iff the eigenvalue~\(\lambda\) is one of the diagonal elements of~\(A\).

As an example of the non-orthogonality of eigenvectors, consider the two eigenvalues \(\lambda_1=a_{1,1}\) and \(\lambda_2=a_{2,2}\) with  corresponding eigenvectors \(\xv_1=(1,0,0,\ldots,0)\) and \(\xv_2=(-a_{1,2}/(a_{1,1}-a_{2,2}),1,0,\ldots,0)\).
Then the dot product \(\xv_1\cdot\xv_2=-a_{1,2}/(a_{1,1}-a_{2,2})\neq 0\) in general (the dot product is zero only when \(a_{1,2}=0\)).
Since the dot product is generally non-zero, \(\xv_1\) and~\(\xv_2\) are generally not orthogonal.
Similarly for other pairs of eigenvectors corresponding to distinct eigenvalues.
\end{proof}



\begin{example} \label{eg:trieig}
Use Theorem~\ref{thm:trieig} to find the eigenvalues, corresponding eigenvectors, and corresponding eigenspaces, of the following triangular matrices.
\begin{enumerate}
\item \(\eAii=\begin{bmatrix}-3&2&0
\\0&-4&2
\\0&0&4\end{bmatrix}\)
\begin{solution} 
Matrix~\(\eAii\)\ is upper triangular so read off the eigenvalues from the diagonal to be \(-3\) and~\(\pm 4\).
\begin{itemize}
\item For \(\lambda=-3\)\,, and by inspection, all eigenvectors are proportional to \((1,0,0)\).
Hence eigenspace \(\EE_{-3}=\Span\{(1,0,0)\}\).
\item For \(\lambda=-4\) we need to solve
\begin{equation*}
(\eAii+4I)\xv=
\begin{bmatrix}1&2&0
\\0&0&2
\\0&0&8\end{bmatrix}\xv=\ov\,.
\end{equation*}
By inspection an eigenvector must be of the form \((x_1,1,0)\).
And the first line of the system then asserts \(x_1+2=0\)\,.
Hence eigenvectors are proportional to \((-2,1,0)\).
That is, the eigenspace \(\EE_{-4}=\Span\{(-2,1,0)\}\).
\item For \(\lambda=+4\) we need to solve
\begin{equation*}
(\eAii-4I)\xv=
\begin{bmatrix}-7&2&0
\\0&-8&2
\\0&0&0\end{bmatrix}\xv=\ov\,.
\end{equation*}
Consider eigenvectors of the form \((x_1,x_2,1)\).
The second line asserts \(-8x_2+2=0\), that is \(x_2=\frac14\).
The first line asserts \(-7x_1+2x_2=0\), that is \(x_1=\frac27x_2=\frac1{14}\).
Hence eigenvectors are proportional to \((\frac1{14},\frac14,1)\).
That is, the eigenspace \(\EE_{4}=\Span\{(\frac1{14},\frac14,1)\}\).
\end{itemize}
\end{solution}

\item \(\eAii=\begin{bmatrix}3&0&0&0
\\-2&-4&0&0
\\-3&1&0&0
\\0&0&-3&1 \end{bmatrix}\)
\begin{solution} 
Matrix~\(\eAii\)\ is lower triangular so read the eigenvalues from the diagonal to be~\(3\), \(-4\), \(0\) and~\(1\).
\begin{itemize}
\item For \(\lambda=1\)\,, by inspection all eigenvectors are of the form \((0,0,0,1)\).
Hence eigenspace \(\EE_{1}=\Span\{(0,0,0,1)\}\).

\item For \(\lambda=0\)\,, seek an eigenvector \((0,0,1,x_4)\) then the last line of the system
\begin{equation*}
(\eAii-0I)\xv=
\begin{bmatrix}3&0&0&0
\\-2&-4&0&0
\\-3&1&0&0
\\0&0&-3&1 \end{bmatrix}\xv=\ov
\end{equation*}
requires \(-3+x_4=0\)\,.  
Hence eigenvectors are proportional to \((0,0,1,3)\).
That is, the eigenspace \(\EE_{0}=\Span\{(0,0,1,3)\}\).

\item For \(\lambda=-4\), seek an eigenvector \((0,1,x_3,x_4)\) then the third line of the system
\begin{equation*}
(\eAii+4I)\xv=
\begin{bmatrix}7&0&0&0
\\-2&0&0&0
\\-3&1&4&0
\\0&0&-3&5 \end{bmatrix}\xv=\ov
\end{equation*}
requires \(1+4x_3=0\)\,, that is \(x_3=-\frac14\).
Then the last line of the system requires \(\frac34+5x_4=0\)\,, that is \(x_4=-\frac3{20}\)\,. 
Hence eigenvectors are proportional to \((0,1,-\frac14,-\frac3{20})\).
That is, the eigenspace \(\EE_{-4}=\Span\{(0,1,-\frac14,-\frac3{20})\}\).

\item For \(\lambda=3\), seek an eigenvector \((1,x_2,x_3,x_4)\) then the second line of the system
\begin{equation*}
(\eAii-3I)\xv=
\begin{bmatrix}0&0&0&0
\\-2&-7&0&0
\\-3&1&-3&0
\\0&0&-3&-2 \end{bmatrix}\xv=\ov
\end{equation*}
requires \(-2-7x_2=0\)\,, that is \(x_2=-\frac27\).
Then the third line of the system requires \(-3-\frac27-3x_3=0\)\,, that is \(x_3=-\frac{23}{21}\)\,. 
Lastly, the last line requires \(\frac{23}7-2x_4=0\)\,, that is \(x_4=\frac{23}{14}\).
Hence eigenvectors are proportional to \((1,-\frac27,-\frac{23}{21},\frac{23}{14})\).
That is, the eigenspace \(\EE_{3}=\Span\{(1,-\frac27,-\frac{23}{21},\frac{23}{14})\}\).

\end{itemize}
\end{solution}

\item \(\eAii=\begin{bmatrix} -1&1&-8&-5&5
\\-3&6&4&-3&0
\\1&-3&1&0&0
\\-7&1&0&0&0
\\-1&0&0&0&0 \end{bmatrix}\)
\begin{solution} 
Matrix~\(\eAii\) is not a triangular matrix (Definition~\ref{def:trim}), so Theorem~\ref{thm:trieig} does not apply.
Row or column swaps could transform it to be triangular, but we have not investigated the effect of such swaps on eigenvalues and eigenvectors.
\end{solution}
\end{enumerate}
\end{example}






One consequence of the second part of the proof of Theorem~\ref{thm:trieig} is that, when counted according to multiplicity, there are precisely \(n\)~eigenvalues of an \(n\times n\) diagonal matrix.
Correspondingly, the next section establishes there are precisely \(n\)~eigenvalues of general \(n\times n\) matrices, provided we count the eigenvalues according to multiplicity and allow complex valued eigenvalues.








\begin{exercise} \label{ex:} 
Each of the following pictures applies to some specific real matrix, say called~\(A\).
The pictures plot~\(A\xv\) adjoined to the end of unit vectors~\xv.
By inspection decide whether the matrix, in each case, has real eigenvalues or \idx{complex eigenvalue}s.
%\begin{verbatim}
%a=round(eye(2)*5+randn(2)*5)/10,eig(a),num2str(a(:)')
%\end{verbatim}

\begin{parts}
\item \eRose{0.2}{-1.1}{ 1.3}{ 0.9}
\answer{complex}

\item \eRose{0.9}{0.9}{0.8}{0.2}
\answer{real}

\item \eRose{0.5}{ 0.3}{-0.8}{ 1.3}
\answer{complex}

\item \eRose{-0.7}{-0.2}{ 0.4}{-0.3}
\answer{complex}

\item \eRose{0.3}{-0.3}{-0.2}{1}
\answer{real}

\item \eRose{0.1}{-0.4}{ 0.8}{ 0.7}
\answer{complex}

\item \eRose{0.8}{-0.7}{-0.3}{ 0.8}
\answer{real}

\item \eRose{0.8}{ 0.3}{-0.6}{1}
\answer{complex}

\end{parts}
\end{exercise}








\begin{exercise} \label{ex:} 
For each of the following \index{triangular matrix}triangular matrices, write down all \idx{eigenvalue}s and then find the corresponding \idx{eigenspace}s.
Show your working.
%\begin{verbatim}
%format rat, n=3
%A=0+tril(round(randn(n)*3)), if rand>0.5, A=A', end, [V,D]=eig(A); DVt=[diag(D) diag(1./max(abs(V)))*V']
%\end{verbatim}
\begin{parts}
\item \(\begin{bmatrix} 2 & 0
\\-1 & 4 \end{bmatrix}\)
\answer{\(\EE_{4}=\Span\{(0,1)\}\), 
\(\EE_{2}=\Span\{(1,\frac12)\}\).}

\item \(\begin{bmatrix} 2 & 0
\\-3 & 2 \end{bmatrix}\)
\answer{\(\EE_{2}=\Span\{(0,1)\}\).}

\item \(\begin{bmatrix} -1 & 0 & 3
\\0 & -1 & 2
\\0 & 0 & -5 \end{bmatrix}\)
\answer{\(\EE_{-1}=\Span\{(1,0,0),(0,1,0)\}\), 
\(\EE_{-5}=\Span\{(-3,-2,4)\}\).}

\item \(\begin{bmatrix} 0 & 0 & 0
\\-3 & -4 & 0
\\1 & 5 & -2 \end{bmatrix}\)
\answer{\(\EE_{-2}=\Span\{(0,0,1)\}\), 
\(\EE_{-4}=\Span\{(0,2,-5)\}\), 
\(\EE_{0}=\Span\{(8,-6,-11)\}\).}

\item \(\begin{bmatrix} 1 & -5 & 0
\\0 & 0 & -4
\\0 & 0 & 0 \end{bmatrix}\)
\answer{\(\EE_{1}=\Span\{(1,0,0)\}\), 
\(\EE_{0}=\Span\{(5,1,0)\}\).}

\item \(\begin{bmatrix} 0 & 0 & -2
\\0 & 4 & 1
\\-3 & -3 & -1 \end{bmatrix}\)
\answer{This matrix is not triangular so we cannot answer (as yet).}

\item \(\begin{bmatrix} -1 & 0 & 0
\\-2 & 2 & 0
\\-2 & -1 & -1 \end{bmatrix}\)
\answer{\(\EE_{-1}=\Span\{(0,0,1)\}\), 
\(\EE_{2}=\Span\{(0,3,-1)\}\).}

\item \(\begin{bmatrix} -2 & 4 & -2 & -2
\\0 & -2 & 1 & 7
\\0 & 0 & -3 & 1
\\0 & 0 & 0 & 2 \end{bmatrix}\)
\answer{\(\EE_{-2}=\Span\{(1,0,0,0)\}\), 
\(\EE_{-3}=\Span\{(6,-1,1,0)\}\), 
\(\EE_{2}=\Span\{(6,9,1,5)\}\).}

\item \(\begin{bmatrix} 8 & -2 & 3 & 2
\\0 & -6 & 1 & -2
\\0 & 0 & 3 & -2
\\0 & 0 & 0 & 0 \end{bmatrix}\)
\answer{\(\EE_{8}=\Span\{(1,0,0,0)\}\), 
\(\EE_{-6}=\Span\{(1,7,0,0)\}\), 
\(\EE_{3}=\Span\{(-5,1,9,0)\}\), 
\(\EE_{0}=\Span\{(-5,-2,6,9)\}\).}

\item \(\begin{bmatrix} 0 & -2 & -5 & 2
\\0 & 7 & -1 & 2
\\0 & 0 & 3 & -4
\\0 & 0 & 0 & 3 \end{bmatrix}\)
\answer{\(\EE_{0}=\Span\{(1,0,0,0)\}\), 
\(\EE_{7}=\Span\{(-2,7,0,0)\}\), 
\(\EE_{3}=\Span\{(-22,3,12,0)\}\).}

\end{parts}
\end{exercise}







\endinput
