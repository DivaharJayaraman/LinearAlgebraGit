%!TEX root = ../larxxia.tex

\chapter{Determinants distinguish matrices}% brand?
\label{ch:ddm}
\minitoc

\begin{comment}
\pooliv{\S4.2} \layiv{\S3.1--2} \holti{\S5.1--2} 
This development is inspired by that of \cite{Hannah96} and the Uni Christchurch (\verb|ChristchurchDeterminants.pdf|).

The typical teaching of determinants starts from our endpoint: it starts with the Laplace Expansion Theorem~\ref{thm:letdet} as a definition of determinants and then works algebraically through the same properties developed here, but roughly in reverse order.
Many might view that the geometric development of the determinant presented here is not as rigorous as the typical algebraic development.
That may well be, but at its heart the determinant expresses a key geometric property of the matrix.
Also, in terms of learning and using determinants this geometric approach appears easier to learn, gets to key practical properties quicker, and better reinforces the properties that make determinants useful in applications.
\end{comment}

 
Although much of the theoretical role of determinants is usurped by the \svd, nonetheless, determinants aid in establishing forthcoming properties of eigenvalues and eigenvectors, and empower graduates to connect to much extant practice.

Recall from previous study (section~\ref{sec:feebh}, e.g.)
\begin{itemize}
\item a \(2\times 2\) matrix \(A=\begin{bmatrix} a&b\\c&d \end{bmatrix}\) has \idx{determinant} \(\det A=|A|=ad-bc\)\,, and matrix~\(A\) is  \idx{invertible} iff \(\det A\neq0\)\,;
\item a \(3\times 3\) matrix \(A=\begin{bmatrix} a&b&c\\d&e&f\\g&h&i \end{bmatrix}\) has \idx{determinant} \(\det A=|A|=aei+bfg+cdh-ceg-afh-bdi\)\,, and matrix~\(A\) is \idx{invertible} iff \(\det A\neq0\)\,.
\end{itemize}
These two formulas for a determinant are best remembered via the following diagrams where products along the red lines are subtracted from the products along the blue lines, respectively:
\begin{equation}
\input{dettwothree.ltx}
\label{eq:dets23b}
\end{equation}
This chapter extends these determinants to any size matrix, and explores more useful properties---especially those properties useful for understanding and developing the general eigenvalue problems and applications of Chapter~\ref{ch:gee}.





\endinput

